%
% Kalenderserien.tex
%
% $Id: Kalenderserien.tex 474 2006-06-07 17:15:29Z danny $
%
% Copyright (c) 1998-2003 AMC World Technologies
% Fischerinsel 4, D-10179 Berlin, Deutschland
% All Rights Reserved.
%
% This software is the confidential and proprietary information of AMC World
% Technologies ("Confidential Information"). You shall not disclose such
% Confidential Information and shall use it only in accordance with the terms
% of the licence agreement you entered into with AMC World Technologies.
%


%================================================
%
%
% Praeambel
%
%
%================================================
\documentclass[a4paper]{article}

%--------------------------------------
%
% Packages
%
%--------------------------------------
\usepackage{amsmath}
\usepackage{amssymb}
\usepackage{amsthm}
\usepackage{ngerman}
\usepackage{tabularx}


%--------------------------------------
%
% Kommandos
%
%--------------------------------------
\newcommand*{\dayf}{\mathrm{day}}
\newcommand*{\monf}{\mathrm{mon}}
\newcommand*{\yearf}{\mathrm{year}}
\newcommand*{\wdf}{\mathrm{wd}}
\newcommand*{\weekf}{\mathrm{week}}
\newcommand*{\womf}{\mathrm{wom}}
\newcommand*{\calf}{\mathrm{cal}}
\newcommand*{\datev}[1]{\texttt{`#1'}}
\newcommand*{\gff}{\mathrm{gf}}
\newcommand*{\sff}{\mathrm{sf}}
\newcommand*{\addff}{\boxplus}
\renewcommand{\theequation}{\thesection.\arabic{equation}}
\numberwithin{equation}{section}
\newtheorem{dfn}{Definition}
\newtheorem{thm}{Satz}


%--------------------------------------
%
% Laengen
%
%--------------------------------------
\newlength{\widthTab}  % Breite von bestimmten Tabellen


%--------------------------------------
%
% Titel
%
%--------------------------------------
\title{Berechnung von Kalenderserien}
\author{Daniel Ellermann}
\date{30. Oktober 2003}


%--------------------------------------
%
% Dokumentstart
%
%--------------------------------------
\begin{document}
\maketitle
\tableofcontents



%================================================
%
%
% Inspiration
%
%
%================================================
\section{Inspiration}
Die Berechnung und Verwaltung von Kalenderserien ist eine z. T. nicht triviale
Aufgabe. sYnergy unterst"utzt die wichtigsten Muster f"ur Terminserien, ohne
dabei den Benutzer unn"otig zu verwirren. Es bestand der Anspruch, f"ur
s"amtliche zur Verf"ugung stehenden Serienmuster eine Formel zur Berechnung
des Zeitpunkts bei einer gegebenen Anzahl $n$ von Terminen der Serie zu haben.

Weiterhin bietet dieser Artikel wertvolle Formeln, die f"ur die
Kalenderberechnung von ausschlaggebender Bedeutung sind.



%================================================
%
%
% Serienmuster
%
%
%================================================
\section{Serienmuster}
Die definierten Serientypen in sYnergy finden Sie in Tabelle \ref{tab:pattern}
auf Seite \pageref{tab:pattern}. Jedes Muster hat die Form

\vspace{8pt}
\texttt{fall d m y wd [ ursprung ]}
\vspace{8pt}

\noindent Dabei stellt \texttt{fall} die Fallnummer nach
Tabelle~\ref{tab:pattern}, Spalte ''Fall'' dar, sowie \texttt{d} das
Tagesmuster, \texttt{m} das Monatsmuster, \texttt{y} das Jahresmuster und
\texttt{wd} das Wochentagsmuster.

Jedes dieser Muster besteht aus einem Basiswert und einem optionalen
Qualifizierer. F"ur jedes der vier Muster (Tagesmuster, Monatsmuster,
Jahresmuster und Wochentagsmuster) muss genau einer der folgenden m"oglichen
Basiswerte angegeben werden:
\begin{itemize}
\item \texttt{*} (beliebiger Wert). Das Feld passt auf jeden beliebigen Wert.
\item \texttt{\textit{x}} (exakter Wert). Das Feld passt dann und nur dann, wenn
  es den Wert \texttt{\textit{x}} hat.
\item \texttt{\textit{a}-\textit{b}} (Bereichswert). Repr"asentiert Werte im
  Bereich von \texttt{\textit{a}} bis \texttt{\textit{b}} (beide
  eingeschlossen).
\item \texttt{\textit{a}, \textit{b}, \ldots} (Aufz"ahlungswert). Repr"asentiert
  die Werte \texttt{\textit{a}}, \texttt{\textit{b}} usw.
\item \texttt{+\textit{x}} bzw. \texttt{-\textit{x}} (Relativwert). Stellt einen
  relativen Wert zu einem gegebenen Ursprung \texttt{\textit{ursprung}} dar, der
  am Ende des Musters angegeben wird. Wird kein Ursprung spezifiziert, wird der
  1. Januar angenommen. Beispiel: Das Muster \texttt{+49 * * * E} mit \texttt{E}
  als Kennzeichnung des Ostersonntags (Eastern) als Ursprung repr"asentiert den
  Pfingstsonntag, der 49 Tage nach dem Ostersonntag liegt. Relativwerte sind nur
  im Tages- und Monatsmuster erlaubt.
\end{itemize}

\noindent Dar"uber hinaus kann jeder dieser Basiswerte mit den folgenden
optionalen Qualifizierern gekennzeichnet werden. In den nachfolgenden
Definitionen steht \texttt{\textit{b}} f"ur einen Basiswert.
\begin{itemize}
\item \texttt{\textit{b}/\textit{n}} (Schrittwert-Qualifizierer). Das Feld passt
  auf jedes \texttt{\textit{n}}. Vorkommen des Wertes \texttt{\textit{b}}. Dabei
  darf \texttt{\textit{b}} allerdings kein Relativwert sein. Beispiel: der Wert
  \texttt{*/2} im Monatsfeld bedeutet aller zwei Monate; der Wert \texttt{1,3/3}
  im Wochentagsfeld bedeutet Montag und Mittwoch aller drei Wochen.
\item \texttt{\textit{b}:\textit{y}} (Ordinalwert). Bedeutet den
  \texttt{\textit{y}}. Wert \texttt{\textit{b}}. Ordinalwerte sind nur im
  Wochentagsfeld erlaubt. M"oglich sind auch negative Werte f"ur
  \texttt{\textit{y}}, wobei diese dann vom Ende eines Monats gerechnet werden.
  Beispiel: der Wert \texttt{4:2} stellt den zweiten Donnerstag dar, der Wert
  \texttt{2:-1} bedeutet den letzten Dienstag im Monat. Momentan werden bei
  Ordinalwerten als zugeh"origer Basiswert nur \texttt{*} und ein exakter Wert
  unterst"utzt.
\end{itemize}

\setcounter{footnote}{1}
\begin{table}[ht]
\caption{Muster f"ur Terminserien}
\label{tab:pattern}\begin{tabularx}{\textwidth}{ccXc}
\hline
Fall & Muster & Beschreibung & Intervall $\varepsilon$ \\
\hline
0 & - & Sonderf"alle, z. B. Feiertage & - \\
100 & \texttt{*/$n$ * * *} & aller $n$ Tage & $n$ Tag(e) \\
200 & \texttt{* * * 1-5} & jeden Arbeitstag & - \\
300 & \texttt{* * * $ws$/$w$} & an den Wochentagen $ws$ aller $w$ Wochen & - \\
400 & \texttt{$d$ */$n$ * *} & am $d$. Tag jedes $n$. Monats & $n$ Monat(e) \\
500 & \texttt{* */$n$ * $ws$:$w$} & am $w$. Wochentag $ws$ jedes $n$. Monats &
  $n$ Monat(e)\footnotemark[\ref{ft:oneMonth}] \\
600 & \texttt{$d$ $m$ * * *} & am $d$. des Monats $m$ (Geburtstagsserie) &
  1 Jahr \\
700 & \texttt{* $m$ * $ws$:$w$} & am $w$. Wochentag $ws$ im Monat $m$
  (Muttertagsserie) & 1 Jahr\footnotemark[\ref{ft:oneMonth}] \\
\hline
\end{tabularx}\end{table}
\footnotetext{\label{ft:oneMonth}Im Monat des Vorkommens des Serienelements ist
  der $w$. Wochentag zu w"ahlen.}

\label{badInterval}
Bei den Intervallen $n$ Monat(e) bzw. 1 Jahr ist zu beachten, dass sich diese
Werte programmatisch schlecht abbilden lassen. Eine Umrechnung in einen
Millisekundenwert ist nicht m"oglich, da die L"ange eines Monats bzw. eines
Jahres variieren kann.



%================================================
%
%
% Voraussetzungen
%
%
%================================================
\section{Voraussetzungen}
In den nachfolgenden Berechnungen seien folgende Definitionen gegeben.


%--------------------------------------
%
% Wochendaten
%
%--------------------------------------
\subsection{Wochendaten}
F"ur Daten die Woche betreffend, definieren wir folgendes:
\begin{dfn}[Wochenl"ange]
Die Anzahl der Tage in einer Woche bezeichnen wir mit $T$. Es gilt:
\begin{equation}T := 7\end{equation}
\end{dfn}

\begin{dfn}[Wochentage]\label{def:weekdays}
  Die Wochentage werden durch Indizes beginnend mit null beschrieben und mit
  $d_w$ bezeichnet. Sie sind wie folgt definiert:
  \par
  \begin{table}[ht]
  \caption{Wochentage}\label{tab:weekDays}
  \begin{tabularx}{\textwidth}{X|ccccccc}
    \hline
    Wochentag & Montag & Dienstag & Mittwoch & Donnerstag \\
    Index $d_w$ & 1 & 2 & 3 & 4 \\
    \hline
    Wochentag & Freitag & Samstag & Sonntag & \\
    Index $d_w$ & 5 & 6 & 0 & \\
    \hline
  \end{tabularx}\end{table}
  \par
  \noindent Es gilt:
  \begin{equation}d_w \in \mathbb{N},\ 0 \le d_w < T\end{equation}
  \par
  \noindent Die Menge aller Wochentage wird mit $D_w$ bezeichnet und es gilt:
  \begin{equation}D_w := \{ x : x = d_w(i) \},\ 0 \le i < T\end{equation}
\end{dfn}

\begin{dfn}[Wochenanfang]
  Der Index des Wochentages des ersten Tags in der Woche wird mit $\eta$
  bezeichnet und es gilt:
  \begin{equation}\eta \in D_w\end{equation}
  In Deutschland gilt $\eta := 1$, in Gro"sbritannien gilt
  $\eta := 0$.
\end{dfn}


%--------------------------------------
%
% Terminserien
%
%--------------------------------------
\subsection{Terminserien}
F"ur die betrachteten Terminserien definieren wir folgendes:

\begin{dfn}[Anzahl der Elemente einer Terminserie]\label{def:numOfElements}
  Die vorgegebene Anzahl von Terminen innerhalb einer Terminserie bezeichnen wir
  mit $n$. Es gilt:
  \begin{equation}n \in \mathbb{N}, n > 0\end{equation}
\end{dfn}

\begin{dfn}[Anfangs- und Endpunkt einer Terminserie]
  Die beiden benutzerdefinierten Anfangs- und Endzeitpunkte der Serie werden mit
  $t_s$ bzw. $t_e$ bezeichnet. Diese Zeitpunkte m"ussen nicht zur Serie selbst
  geh"oren.
\end{dfn}

\begin{dfn}[Kalibrierter Startpunkt]
  Der kalibrierte Startpunkt einer Terminserie wird mit $t'_s$ bezeichnet. Dies
  kann entweder das Startdatum der Terminserie sein, oder es handelt sich dabei
  um einen Betrachtungszeitpunkt. In jedem Fall muss $t'_s$ selbst ein Element
  der Serie sein.
\end{dfn}

\begin{dfn}[Terminserie als Folge]\label{def:seriesAsSeries}
  Eine Terminserie stellt eine Folge $s_n$ von Serienelementen dar, f"ur die
  gilt:
  \begin{eqnarray}
    s_1 & := & t'_s \\
    s_{i+1} & := & s_i + \varepsilon
  \end{eqnarray}
\end{dfn}

Manchmal ben"otigen wir auch die Serienelemente in der Mengenschreibweise und
beschreiben das durch folgende Definition:
\begin{dfn}[Serienelementmenge]
  Die geordnete Menge aller Elemente einer Serie bezeichnen wir als $\vec{S}$
  und es gilt:
  \begin{equation}
    \vec{S} := \{ x : x = s_i \} \text{ mit } 0 < i \le n
  \end{equation}
\end{dfn}


%--------------------------------------
%
% Mengen
%
%--------------------------------------
\subsection{Mengen}
Es seien die folgenden Mengen definiert:
\begin{dfn}
  Die Menge der Tage $M_d$ in einem Monat ist wie folgt definiert:
  \begin{equation}
    M_d := \{ q : q \in \mathbb{N},\ 1 \le q \le 31 \}
  \end{equation}
\end{dfn}

\begin{dfn}
  Die Menge der Monate $M_m$ in einem Jahr ist wie folgt definiert:
  \begin{equation}
    M_m := \{ q : q \in \mathbb{N},\ 1 \le q \le 12 \}
  \end{equation}
\end{dfn}

\begin{dfn}
  Die Menge der Jahre $M_y$ ist wie folgt definiert:
  \begin{equation}M_y := \mathbb{Z} \backslash 0\end{equation}
\end{dfn}

\begin{dfn}
  Die Menge der Wochen $M_w$ in einem Jahr ist wie folgt definiert:
  \begin{equation}
    M_w := \{ q : q \in \mathbb{N},\ 1 \le q \le 53 \}
  \end{equation}
\end{dfn}

\begin{dfn}
  Die Menge der Wochen $M_{wom}$ in einem Monat ist wie folgt definiert:
  \begin{equation}
    M_{wom} := \{ q : q \in \mathbb{N},\ 1 \le q \le 5 \}
  \end{equation}
\end{dfn}

\noindent Die Menge der Wochentage wurde bereits in
Definition~\ref{def:weekdays} auf Seite~\pageref{def:weekdays} festgelegt.


%--------------------------------------
%
% Funktionen
%
%--------------------------------------
\subsection{Funktionen}
Wir ben"otigen eine Funktion, die den Index eines Wochentages $x \in D_w$ in
einer geordneten Menge $\vec{M}$ von Wochentagsindizes sucht. Dabei gelte
$\forall\ p \in \vec{M} : p \in D_w$. Folgende Definition soll das liefern:
\begin{dfn}
\begin{equation}
  p(\vec{M}, x) := \left\{\begin{array}{ll}
  \exists\ q_i \in \vec{M} \textrm{ mit } q_i = x : & i \\
  \nexists\ q_i \in \vec{M} \textrm{ mit } q_i = x : & 0 \\
  \end{array}\right.,\quad
  1 \le i \le ||\vec{M}||
\end{equation}
\end{dfn}
\noindent Beispiel: sei $\vec{M} = \{2, 4, 5, 6\}$, dann gilt
$p(\vec{M}, 5) = 3$ und $p(\vec{M}, 1) = 0$.

Es seien eine Reihe von Funktionen gegeben, die bestimmte Elemente eines
gegebenen Datums $d$ extrahieren.
\begin{dfn}[Datumsfunktionen]
  Zur Extraktion eines bestimmten Kalendarelements eines gegebenen Datums $d$
  sind die Funktionen in Tabelle~\ref{tab:dateFunctions} auf
  Seite~\pageref{tab:dateFunctions} definiert.
  \begin{table}[tp]
  \caption{Datumsfunktionen}\label{tab:dateFunctions}
  \begin{tabularx}{\textwidth}{cXl}
  \hline
  Funktion & Beschreibung & Beispiel \\
  \hline
  $\dayf(d)$ & berechnet den Tag von $d$. Es gilt $\dayf(d) \in M_d$. &
    $\dayf(d) = 13$ \\
  $\monf(d)$ & berechnet den Monat von $d$. Es gilt $\monf(d) \in M_m$. &
    $\monf(d) = 5$ \\
  $\yearf(d)$ & berechnet das Jahr von $d$. Es gilt $\yearf(d) \in M_y$ &
    $\yearf(d) = 2002$. \\
  $\wdf(d)$ & berechnet den Wochentagsindex von $d$. Es gilt $\wdf(d) \in D_w$. &
    $\wdf(d) = 1$ \\
  $\weekf(d)$ & berechnet die Wochennummer von $d$. Es gilt $\weekf(d) \in M_w$.
    & $\weekf(d) = 20$ \\
  $\womf(d)$ & berechnet die Wochennummer von $d$ innerhalb eines Monats. Es
    gilt: $\womf(d) \in M_{wom}$. & $\womf(d) = 3$ \\
  \hline
  \multicolumn{3}{c}{Die Beispiele setzen $d = \datev{13.05.2002}$ voraus.}
  \end{tabularx}\end{table}
\end{dfn}

Weiterhin brauchen wir eine Funktion, die aus angegebenen Kalenderelementen das
zugeh"orige Datum errechnet:
\begin{dfn}[Kalenderfunktion]
  Die Funktion
  \begin{equation*}\calf(d, m, y, wd, week, wom)\end{equation*}
  berechnet aus den gegebenen, optionalen Elementen Tag im Monat $d \in M_d$,
  Monat $m \in M_m$, Jahr $y \in M_y$, Wochentag $wd \in D_w$, Wochennummer
  $week \in M_w$ und Wochenummer innerhalb eines Monats $wom \in M_{wom}$ das
  zugeh"orige Datum. Dabei k"onnen Elemente, die nicht von Bedeutung sind,
  weggelassen werden, wobei die Anzahl der Kommata unver"anderlich bleiben muss.
  Es gilt:
  \begin{multline}
  q = \calf(d, m, y, wd, week, wom) \Leftrightarrow \\
    (\exists\ d \Rightarrow \dayf(q) = d)\ \wedge \\
    (\exists\ m \Rightarrow \monf(q) = m)\ \wedge \\
    (\exists\ y \Rightarrow \yearf(q) = y)\ \wedge \\
    (\exists\ wd \Rightarrow \wdf(q) = wd)\ \wedge \\
    (\exists\ week \Rightarrow \weekf(q) = week)\ \wedge \\
    (\exists\ wom \Rightarrow \womf(q) = wom) \\
  \end{multline}
\end{dfn}
\noindent Beispiel: $\calf(13, 5, 2002, , ,) = \datev{13.05.2002}$.

Zum allgemeinen Arbeiten mit Kalenderfeldern wollen wir noch zwei Funktionen
definieren, die den Wert eines Kalenderfeldes liefern und setzen, sowie eine
Funktion zum Ver"andern eines Kalenderfeldes.
\begin{dfn}[lesender Kalenderfeldzugriff]
  Die Funktion $\gff$ (f"ur \emph{get calendar field}) ermittelt den Wert eines
  Kalenderfeldes $f$ in einem Datum $d$. Sie ist wie folgt definiert:
  \begin{equation}
    \gff(f, d) := f(d),\ f \in \{\dayf, \monf, \yearf, \wdf, \weekf, \womf\}
  \end{equation}
\end{dfn}
\noindent Beispiel: $\gff(\monf, \datev{13.05.2002}) = 5$.

\begin{dfn}[schreibender Kalenderfeldzugriff]
  Die Funktion $\sff$ (f"ur \emph{set calendar field}) setzt den Wert eines
  Kalenderfeldes
  \begin{equation*}
    f \in \{\dayf, \monf, \yearf, \wdf, \weekf, \womf\}
  \end{equation*}
  in einem Datum $d$ auf den Wert $q$. Sie ist wie folgt definiert:
  \begin{equation}
    \sff(f, d, q) := \left\{\begin{array}{ll}
    f = \dayf : & \calf(q, \monf(d), \yearf(d), , , ) \\
    f = \monf : & \calf(\dayf(d), q, \yearf(d), , , ) \\
    f = \yearf : & \calf(\dayf(d), \monf(d), q, , ,) \\
    f = \wdf : & \calf(, \monf(d), \yearf(d), q, \weekf(d), ) \\
    f = \weekf : & \calf(, \monf(d), \yearf(d), \wdf(d), q, ) \\
    f = \womf : & \calf(, \monf(d), \yearf(d), \wdf(d), , q) \\
    \end{array}\right.
  \end{equation}
\end{dfn}
\noindent Beispiel: $\sff(\monf, \datev{13.05.2002}, 8) = \datev{13.08.2002}$.

\begin{dfn}["andernder Kalenderfeldzugriff]
  Die Funktion $\addff$ "andert den Wert eines Kalenderfeldes $f \in \{\dayf,
  \monf, \yearf, \wdf, \weekf, \womf\}$ in einem Datum $d$ um den Wert $q$. Sie
  ist wie folgt definiert:
  \begin{equation}
    \addff(f, d, q) := \sff(f, d, \gff(f, d) + q)
  \end{equation}
\end{dfn}


%--------------------------------------
%
% Wichtige Feststellungen
%
%--------------------------------------
\subsection{Wichtige Feststellungen}
Resultierend aus den vorhergehenden Definitionen k"onnen wir eine Reihe von
Feststellungen notieren.

\begin{thm}[Ende einer Terminserie]\label{thm:endOfSeries}
  Das Ende einer Terminserie $t_e$ ergibt sich wie folgt:
  \begin{equation}t_e := s_n\end{equation}
\end{thm}
\begin{proof}[Beweis]
  Nach Definition~\ref{def:numOfElements} ist $n$ die Anzahl der Termine in der
  Serie, d. h. die Serie endet nach $n$ Serienelementen. Da wir nach
  Definition~\ref{def:seriesAsSeries} eine Serie als Folge auf\mbox{}fassen,
  muss das letzte Element $s_n$ sein.
\end{proof}

\begin{thm}[Unechte Terminserie]
  Eine Terminserie mit nur einem Element bezeichnen wir als ``unechte
  Terminserie''. Es gilt:
  \begin{equation}t'_s = t_e \Leftrightarrow n = 1\end{equation}
\end{thm}
\begin{proof}[Beweis]
  Wir beweisen zuerst $t'_s = t_e \Rightarrow n = 1$. Laut
  Definition~\ref{def:seriesAsSeries} ist $s_1 = t'_s$. Wenn nun gilt
  $t'_s = t_e$, dann muss auch $t_e = s_1$ gelten, d. h. $s_n$ mit $n = 1$.
  \par
  Nun beweisen wir $n = 1 \Rightarrow t'_s = t_e$. Wenn $n = 1$, dann muss
  $s_n = s_1 = t'_s$ gelten nach Definition~\ref{def:seriesAsSeries}. Nun gilt
  aber auch $s_n = s_1 = t_e$ nach Satz~\ref{thm:endOfSeries}. Daraus folgt
  $t'_s = t_e$.
\end{proof}

Die Betrachtung der Serienmuster 200 und 300 ist gleich, wenn man den folgenden
Zusammenhang ber"ucksichtigt:
\begin{equation}
  \label{eqn:200to300}\texttt{* * * 1-5}\ \hat{=}\ \texttt{* * * 1,2,3,4,5/1}
\end{equation}
Wir k"onnen uns damit auf die Betrachtung von Fall 300 beschr"anken. Die
angegebenen Wochentage $ws$ k"onnen wir als eine geordnete Menge $\vec{M}$
auf\mbox{}fassen, die den Index jedes Wochentags aus $ws$ enth"alt. Die Angabe
$ws$ kann sowohl in aufz"ahlender als auch in Bereichsschreibweise gemacht
werden. Die Wandlung der aufz"ahlende Schreibweise in die Menge $\vec{M}$ ist
trivial. F"ur die Bereichsschreibweise $a$-$b$ gilt:
\begin{equation}\label{eqn:range}
  \vec{M} = \{ p\,|\ a \le p \le b \}
\end{equation}
Um den Wochenbeginn zu ber"ucksichtigen, muss $\vec{M}$ noch in eine kanonische
Form $\vec{\Theta}$ "uberf"uhrt werden:
\begin{eqnarray}
  \vec{\Theta} & = & \forall\ i,\ 1 \le i \le ||\vec{M}||:
  \left\{\begin{array}{ll}
  \vec{M}_i < \eta : & \vec{\Theta}_i = \vec{M}_i + T \\
  \vec{M}_i \ge \eta : & \vec{\Theta}_i = \vec{M}_i \\
  \end{array}\right. \\
  k & = & ||\vec{\Theta}|| = ||\vec{M}||
\end{eqnarray}
Dadurch (da $\vec{\Theta}$ geordnet ist) entsteht eine Menge mit Wochentagen,
die alle gr"o"ser oder gleich dem Wochenbeginn $\eta$ sind. Au"serdem haben wir
$k$ ermittelt, das die Anzahl der Wochentage in der Menge $\vec{\Theta}$, also
die Kardinalit"at von $\vec{\Theta}$ bezeichnet.



%================================================
%
%
% Pruefung auf ein Serienelement
%
%
%================================================
\section{Pr"ufung auf ein Serienelement}
\noindent Dieser Abschnitt beschreibt, wie gepr"uft werden kann, ob ein
gegebenens Datum $d$ Element einer Terminserie $\vec{S}$ ist. Dazu wird in
folgender Reihenfolge vorgegangen:
\begin{enumerate}
\item Pr"ufung des Basiswertes jedes Musterfeldes in folgender Reihenfolge:
  \begin{enumerate}
  \item Pr"ufung von einfachen Werten, d. h. \texttt{*} und exakten Werten
  \item Pr"ufung von Aufz"ahlungs- und Bereichswerten
  \item Pr"ufung von Relativwerten
  \end{enumerate}
  Anschlie"send sollte jedes Musterfeld durch eine der o. g. Pr"ufungen
  abgedeckt worden sein.
\item Pr"ufung von optionalen Qualifizierern in folgender Reihenfolge:
  \begin{enumerate}
  \item Pr"ufung von Schrittwerten
  \item Pr"ufung von Ordinalwerten
  \end{enumerate}
\end{enumerate}


\subsection{Pr"ufung von Basiswerten}
\subsubsection{Pr"ufung einfacher Werte}
Einfache Werte, d. h. \texttt{*} und exakte Werte werden nach folgenden Regeln
gepr"uft:
\begin{enumerate}
\item Ist der Basiswert \texttt{*} passt er auf jeden beliebigen Wert.
\item Ist der Basiswert der exakte Wert \texttt{\textit{x}}, passt das jeweilige
  Feld von $d$ dann und nur dann, wenn gilt:
  \begin{itemize}
  \item Im Tagesmuster: $\dayf(d) = x$
  \item Im Monatsmuster: $\monf(d) = x$
  \item Im Jahresmuster: $\yearf(d) = x$
  \item Im Wochentagsmuster: $\wdf(d) = x$
  \end{itemize}
\end{enumerate}

\subsubsection{Pr"ufung von Aufz"ahlungs- und Bereichswerten}
Nach~(\ref{eqn:range}) wandeln wir zun"achst Bereichswerte in Aufz"ahlungswerte
um und arbeiten hinfort mit einer Menge $\vec{M}$ von Werten. Das jeweilige Feld
von $d$ passt dann und nur dann, wenn gilt:
\begin{itemize}
\item Im Tagesmuster: $\dayf(d) \in \vec{M}$
\item Im Monatsmuster: $\monf(d) \in \vec{M}$
\item Im Jahresmuster: $\yearf(d) \in \vec{M}$
\item Im Wochentagsmuster: $\wdf(d) \in \vec{M}$
\end{itemize}

\subsubsection{Pr"ufung von Relativwerten}
Zun"achst muss das Vorkommen des Ursprungs $d_U$ im Jahr $\yearf(d)$ ermittelt
werden. Das jeweilige Feld von $d$ passt dann und nur dann, wenn gilt:
\begin{itemize}
\item Im Tagesmuster: $d = \calf(\dayf(d_U) + x, \monf(d_U), \yearf(d_U), , ,)$
\item Im Monatsmuster: $d = \calf(\dayf(d_U), \monf(d_U) + x, \yearf(d_U), , ,)$
\end{itemize}


\subsection{Pr"ufung von Qualifizierern}
Sind Qualifizierer angegeben, werden die gepr"uften Basiswerte damit verkn"upft
und gegen $d$ gepr"uft.

\subsubsection{Pr"ufung von Schrittwerten}
In Abh"angigkeit vom Musterfeld werden die Schrittwerte wie folgt gepr"uft:
\begin{itemize}
\item Tagesmuster. Zun"achst wird die Differenz $\Delta$ zwischen $d$ und $t'_s$
  in Tagen ermittelt:
  \begin{equation}
    \Delta = \left[\frac{d - t'_s}{1 \textrm{ Tag}}\right]
  \end{equation}
  Das Datum $d$ passt dann und nur dann wenn gilt:
  \begin{equation}(\Delta \ge 0) \wedge (\Delta \mod n = 0)\end{equation}
\item Monatsmuster. Nach~(\ref{eqn:monthDiff}) wird zun"achst die Differenz
  $\Delta_m$ in Monaten zwischen $d$ und $t'_s$ ermittelt. Das Datum $d$ passt
  dann und nur dann wenn gilt:
  \begin{equation}(\Delta_m \ge 0) \wedge (\Delta_m \mod n = 0)\end{equation}
\item Jahresmuster. Die Differenz in Jahren $\Delta$ wird zun"achst wie folgt
  errechnet:
  \begin{equation}\Delta = \yearf(d) - \yearf(t'_s)\end{equation}
  Das Datum $d$ passt dann und nur dann wenn gilt:
  \begin{equation}(\Delta \ge 0) \wedge (\Delta \mod n = 0)\end{equation}
\item Wochentagsmuster. Zun"achst wird die Differenz $\Delta$ zwischen $d$ und
  $t'_s$ in Wochen ermittelt:
  \begin{eqnarray}
    \Delta & = & \frac{d - t'_s}{1 \textrm{ Woche}} \\
    \Delta' & = & \left[\Delta\right]
  \end{eqnarray}
  Das Datum $d$ passt wenn
  \begin{equation}\label{eqn:weekStepMatch}
    (\Delta' \ge 0) \wedge (\Delta' \mod n = 0)
  \end{equation}
  (\ref{eqn:weekStepMatch}) gilt allerdings nicht unbedingt, wenn $\Delta' = 0$.
  Dieser Fall kann n"amlich sowohl auftreten, wenn $d = t'_s$ gilt (dann passt
  $d$) als auch wenn $0 < \Delta < 1$ gilt (dann liegt $d$ in der ersten Woche
  nach $t'_s$, also zwischen $t'_s$ und $t'_s + 7$ Tage (und $d$ passt dann
  nicht).
\end{itemize}

\subsubsection{Pr"ufung von Ordinalwerten}
Ordinalwerte k"onnen laut Definition nur in den Wochentagsmustern vorkommen.
Hier ist eine Unterscheidung zwischen $y < 0$ und $y \ge 0$ notwendig:
\begin{itemize}
\item $y \ge 0$. $d$ passt dann und nur dann wenn gilt:
  \begin{equation}\womf(d) = y\end{equation}
\item $y < 0$. Hier erstellen wir uns ein verl"aufiges Datum $d'$,
  da $\forall\ d: \womf(d) \ge 0$ gilt.
  \begin{equation}d' := \calf(, m', \yearf(d), x, , y)\end{equation}
  wobei wir den Monat $m'$ in Abh"angigkeit vom verwendeten Basiswert des
  Monatsmusters $patt$ abh"angig machen:
  \begin{equation}
  m' := \left\{\begin{array}{ll}
  patt = \texttt{*} : & \monf(d) \\
  patt = \texttt{\textit{x}} : & x \\
  \end{array}\right.
  \end{equation}
  Das Datum $d$ passt dann und nur dann wenn gilt:
  \begin{equation}
    (\dayf(d) = \dayf(d')) \wedge (\monf(d) = \monf(d')) \wedge
    (\yearf(d) = \yearf(d'))
  \end{equation}
\end{itemize}



%================================================
%
%
% Kalibrierung des Startzeitpunkts
%
%
%================================================
\section{Kalibrierung des Startzeitpunkts}
\noindent Eine Kalibrierung des Startzeitpunkts einer Terminserie muss immer
dann stattfinden, wenn der vom Benutzer gew"ahlte Serienbeginn nicht selbst Teil
der Serie ist. In anderen Worten: es ist eine Kalibrierung notwendig, wenn gilt:
\begin{equation}
  t_s \notin \vec{S}
\end{equation}
Offensichtlich ist im Fall 100 keine explizite Kalibrierung notwendig, da
\begin{equation}
  \forall\ t_s : t_s \in \vec{S}
\end{equation}
Im Fall 200 muss eine Kalibrierung vorgenommen werden, wenn der Wochentag des
Startzeitpunkts kein Werktag ist, d. h. wd$(t_s) < 1$ oder wd$(t_s) > 5$. In
diesem Fall ist als kalibrierter Startzeitpunkt der Montag der n"achsten Woche
zu verwenden. Dies kann aber, wie in den Voraussetzungen bereits beschrieben,
auch durch die Berechnungen von Fall 300 ermittelt werden.

Die Kalibrierung im Fall 300 ist nicht trivial. Zun"achst ist zu "uberpr"ufen,
ob sich einer der Wochentag $w \in \vec{M}$ noch in der aktuellen Woche
befindet, also die folgende Aussage wahr ist:
\begin{equation}
  \exists\ \xi,\ \wdf(t_s) < \xi < \eta + T :\quad \xi \in \vec{M}
\end{equation}
Ist sie wahr, ist das zugeh"orige Datum das kalibrierte Startdatum:
\begin{equation}
  t'_s = \calf(, , \yearf(t_s), \xi \mod T + 1, \weekf(t_s) +
  \left[\frac{\xi}{T}\right], )
\end{equation}
Andernfalls muss die Suche nach dem passenden Wochentag nach $w - 1$ Wochen
stattfinden:
\begin{equation}
  \exists\ \xi,\ \eta \le \xi < \wdf(t_s) :\quad \xi \in \vec{M}
\end{equation}
Dann ergibt sich als kalibriertes Startdatum:
\begin{equation}
  t'_s = \calf(, , \yearf(t_s), \xi \mod 7 + 1, \weekf(t_s) + w, )
\end{equation}

\noindent Im Fall 400 findet sich das kalibrierte Startdatum wie folgt:
\begin{equation}
  t'_s = \left\{\begin{array}{ll}
  \dayf(t_s) < d: & \calf(d, \monf(t_s), \yearf(t_s), , , ) \\
  \dayf(t_s) > d: & \calf(d, \monf(t_s) + n, \yearf(t_s), , , ) \\
  \end{array}\right.
\end{equation}

\noindent "Ahnlich auch im Fall 500. Sei zun"achst der n"achste infragekommende
Termin $t_x$ mit $t_x = \calf(, \monf(t_s), \yearf(t_s), , ws, w)$ bezeichnet.
Dann gilt f"ur das kalibrierte Startdatum:
\begin{equation}
  t'_s = \left\{\begin{array}{ll}
  t_s < t_x : & t_x \\
  t_s > t_x : & \calf(, \monf(t_s) + n, \yearf(t_s), , ws, w) \\
  \end{array}\right.
\end{equation}

\noindent F"ur Fall 600 gilt unter Ber"ucksichtigung des n"achsten
infragekommenden Termins $t_x$ mit $t_x = \calf(d, m, \yearf(t_s), , , )$:
\begin{equation}
t'_s = \left\{\begin{array}{ll}
t_s < t_x : & t_x \\
t_s > t_x : & \calf(d, m, \yearf(t_s) + 1, , , ) \\
\end{array}\right.
\end{equation}

\noindent Schlie"slich berechnet sich das kalibrierte Startdatum f"ur Fall 700
mit dem n"achsten infragekommenden Termin $t_x$ mit
$t_x = \calf(, m, \yearf(t_s), , ws, w)$ wie folgt:
\begin{equation}
t'_s = \left\{\begin{array}{ll}
t_s < t_x : & t_x \\
t_s > t_x : & \calf(, m, \yearf(t_s) + 1, , ws, w) \\
\end{array}\right.
\end{equation}



%================================================
%
%
% Intervall- und Endpunktberechnung
%
%
%================================================
\section{Intervall- und Endpunktberechnung}
\noindent Wie oben dargestellt, ergibt sich f"ur die F"alle 100 und 400 bis 700
ein regelm"a"siges Intervall $\varepsilon$, und es gilt f"ur die Berechnung des
Zeitpunkts $t_e$ nach $n$ Terminen:
\begin{equation}\label{eqn:endEasy}
  t_e = t'_s + \varepsilon(n - 1)
\end{equation}

\noindent Eine kleine Abweichung stellt Fall 500 dar. Hier wird zun"achst auch
nach~(\ref{eqn:endEasy}) der Monat des Endzeitpunktes berechnet, wobei als
Interval $\varepsilon$ der Wert $n$ Monat(e) angenommen wird. $t_e$ ergibt sich
dann wie folgt:
\begin{equation}t_e = \calf(, \monf(t_e), \yearf(t_e), ws, , w)\end{equation}

\noindent In der gleichen Weise arbeitet auch die Abweichung in Fall 700, jedoch
wird hier f"ur $\varepsilon$ der Wert 1 Jahr angenommen und $t_e$ ergibt sich
nach:
\begin{equation}t_e = \calf(, m, \yearf(t_e), ws, , w)\end{equation}

\noindent Nicht trivial ist dagegen die Berechnung der Intervalle f"ur die
F"alle 200 und 300, und damit auch des Zeitpunkts $t_e$. Auf unsere
transformierte Menge $\vec{\Theta}$ wenden wir eine Funktion $\Delta$ an, die
den Abstand zwischen dem Wochentag an der Stelle $i$ und dem Wochentag an der
Stelle $i+1$ ermittelt:
\begin{equation}
\Delta(\vec{\Theta}, i, w) = \left\{\begin{array}{ll}
1 \le i < k : & \vec{\Theta}_{i+1} - \vec{\Theta}_i \\
i = k : & tw + \vec{\Theta}_{1} - \vec{\Theta}_{i} \\
\end{array}\right., 1 \le i \le k
\end{equation}
Die Funktion $\Delta$ berechnet au"serdem den Abstand zwischen dem letzten
Wochentag $\vec{\Theta}_k$ und dem ersten Wochentag $\vec{\Theta}_1$ nach $w$
Wochen. Der Wert $w$ ergibt sich aus Tabelle \ref{tab:pattern} auf Seite
\pageref{tab:pattern}.

Dadurch entsteht eine Reihe $s_n$ von Terminen dieser Serie:
\begin{eqnarray}
s_1 & = & t'_s \\
s_{j} & = & s_{j-1} + \Delta(\vec{\Theta}, p(\vec{\Theta}, \textrm{wd}(s_{j-1})), w)
\end{eqnarray}

\noindent F"ur eine Anzahl von $n$ Terminen ergibt sich somit folgender
Endtermin $t_e$ im Falle 300:
\begin{equation}
\label{eqn:sn300}t_e = s_n
\end{equation}



%================================================
%
%
% Approximierung
%
%
%================================================
\section{Approximierung}
Eine weitere wichtige Problemstellung ist die Findung eines Termins innerhalb
einer Serie, der einem gegebenen Datum $d_x$ chronologisch als n"achstes folgt.
Die folgende Definition beschreibt das:
\begin{dfn}[approximierter Zeitpunkt]\label{def:approx}
  Gegeben sei ein Datum $d_x$ und eine Terminserie $s$. Unter der Voraussetzung,
  dass $d_x \le t_e = s_n$, gilt
  \begin{equation}\exists\ \vartheta : \vartheta \ge d_x\end{equation}
  Dabei ist $\vartheta$ ein Serienelements $s_i \in \vec{S}$, das $d_x$
  chronologisch als n"achstes folgt:
  \begin{gather}\label{eqn:approxGeneral}
    \vartheta := s_i \textnormal{ mit } d_x = s_i - \delta,\ 1 \le i \le n
      \textnormal{ und} \\
    \delta \ge 0 \textnormal{ und } \lim \delta = 0
  \end{gather}
  Weiterhin gilt:
  \begin{eqnarray}\label{eqn:approxEasy}
    d_x < t'_s \Rightarrow \vartheta = t'_s
  \end{eqnarray}
\end{dfn}

\paragraph{Anmerkungen}
\begin{itemize}
\item Die Voraussetzung $d_x \le t_e = s_n$ ergibt sich daraus, dass abgelaufene
  Termine nie wieder stattfinden werden.
\item Der Wert $\delta$ bestimmt einen m"oglichst kleinen Wert, der, zu $d_x$
  addiert, ein Element $s_i$ der Terminserie ergibt.
\end{itemize}

\noindent Die Ermittlung von $\vartheta$ ist vom Fall der Terminserie abh"angig.
Nachfolgend wird die Berechnung f"ur jeden Fall beleuchtet. Im
Fall~\eqref{eqn:approxEasy} ist $\vartheta$ bereits bestimmt worden. In den
folgenden Abschnitten wollen wir deshalb von $d_x \ge t'_s$ ausgehen. Dazu
halten wir zun"achst folgendes fest:
\begin{thm}\label{thm:dBetweenElems}
  Unter der Voraussetzung $t'_s \le d_x \le t_e$ gilt
  \begin{equation}
    \exists\ i \in \mathbb{N},\ 1 \le i < n :\ s_i \le d_x \le s_{i+1}
  \end{equation}
  $d_x$ muss sich also zwischen zwei benachbarten Serienelementen befinden.
\end{thm}
\begin{proof}[Beweis]
  Vom Bereich her ist Satz~\ref{thm:dBetweenElems} korrekt, denn es gilt
  \begin{gather}
    \forall\ i \in \mathbb{N},\ 1 \le i < n :\ s_i \le d_x \le s_{i+1} \\
    \Leftrightarrow s_1 \le d_x \le s_n \\
    \Leftrightarrow t'_s \le d_x \le t_e
  \end{gather}
  Das wurde durch die Voraussetzung bereits sichergestellt. Nach
  Definition~\ref{def:seriesAsSeries} gilt
  \begin{eqnarray}
    s_{n+1} &=& s_n + \varepsilon \\
    s_{n+1} - s_n &=& \varepsilon
  \end{eqnarray}
  Nach den Werten f"ur $\varepsilon$ aus Tabelle~\ref{tab:pattern} auf
  Seite~\pageref{tab:pattern} erkennen wir, dass $\varepsilon \ge 1$ Tag gilt,
  d. h. $s_{n+1} - s_n \ge 1$ Tag. Die L"ange eines Datums $d_x$ ist
  offensichtlich 1 Tag. Es gilt also
  \begin{equation}s_n + 1 \textrm{ Tag} \le s_n + \varepsilon\end{equation}
\end{proof}


%--------------------------------------
%
% Fall 100
%
%--------------------------------------
\subsection{Fall 100}
Dem Fall 100 liegt als einzigem Fall ein fester, deterministischer Intervall
zugrunde. Wir wollen nun den Index $q$ desjenigen Serienelements $s_q$
ermitteln, f"ur das gilt $d_x = s_q - \delta$ mit $\delta \ge 0$ und
$\lim \delta = 0$. Das bedeutet, wir ermitteln den Index des Serienelements, das
$d_x$ chronologisch als n"achstes folgt. Durch Einsetzen von $q$ in
Gleichung~\eqref{eqn:endEasy} erhalten wir dann $\vartheta$.

Um $q$ zu bestimmen, ersetzen wir in~\eqref{eqn:endEasy} $n$ durch $q$ und $t_e$
durch $d_x$ und stellen nach $q$ um. Wir erhalten:
\begin{eqnarray}
  d_x & = & t'_s + \varepsilon(n - 1) \\
  d_x - t'_s & = & \varepsilon(n - 1) \\
  \frac{d_x - t'_s}{\varepsilon} & = & n - 1 \\
  \frac{d_x - t'_s}{\varepsilon} + 1 & = & n
\end{eqnarray}
Wir k"onnen davon ausgehen, dass $\varepsilon \ge 0$ gilt. Mit
\begin{equation}\label{eqn:nv}n_v = \left[n + \gamma\right]\end{equation}
erhalten wir den Index $n_v$ desjenigen Termins, der chronologisch auf $d_x$
folgt. F"ur $\gamma$ w"ahlen wir einen Wert, der fast 1 ist. Damit erreichen
wir, dass f"ur $n_v$ gilt:
\begin{equation}\label{eqn:nv2}
  n_v = \left\{\begin{array}{ll}
  n - [n] = 0 : & n \\
  n - [n] \ne 0 : & [n] + 1 \\
  \end{array}\right.
\end{equation}
Wir setzen also $\gamma$ wie folgt, wobei wir m"oglichst hohe Werte f"ur $x$
annehmen:
\begin{equation}\label{eqn:gamma}
  \gamma := \lim_{x \to \infty} (1 - \frac{1}{x})
\end{equation}
\begin{thm}
F"ur $\gamma$ gilt:
\begin{equation}1 > \gamma \ge 1 - n + [n]\end{equation}
\end{thm}
\begin{proof}[Beweis]
Der Teil $\gamma < 1$ ergibt sich aus~(\ref{eqn:gamma}):
\begin{eqnarray}
  \gamma & = & \lim_{x \to \infty} (1 - \frac{1}{x}) \\
    & = & \lim_{x \to \infty} 1 - \lim_{x \to \infty} \frac{1}{x} \\
    & = & 1 - 0 \\
    & = & 1
\end{eqnarray}
Den zweiten Teil $\gamma \ge 1 - n + [n]$ beweisen wir durch den Umkehrschluss.
W"are $\gamma < 1 - n + [n]$ wahr, gilt $\gamma = 1 - n + [n] - \delta$ mit
$\delta > 0$. Dann gilt nach~(\ref{eqn:nv}) f"ur den Fall $n - [n] \ne 0$:
\begin{eqnarray}
  n_v & = & [ n + \gamma ] \\
    & = & [ n + 1 - n + [n] - \delta ] \\
    & = & [ 1 + [n] - \delta ]
\end{eqnarray}
Dies wiederspricht jedoch~(\ref{eqn:nv2}), wo bei $n - [n] \ne 0$ festgelegt
wurde $n_v = [n] + 1$.
\end{proof}
\noindent Wir erhalten $\vartheta$ durch~\eqref{eqn:endEasy}, wobei wir $t_e$
durch $\vartheta$ und $n$ durch $n_v$ ersetzen:
\begin{equation}\vartheta = t'_s + \varepsilon (n_v-1)\end{equation}


%--------------------------------------
%
% Faelle 200 und 300
%
%--------------------------------------
\subsection{F"alle 200 und 300}
Um $\vartheta$ f"ur die F"alle 200 und 300 zu berechnen, ist eine umfangreichere
Betrachtung vonn"oten. Auch hier "uberf"uhren wir Fall 200 nach
Satz~\ref{eqn:200to300} in den Fall 300 (siehe Seite~\pageref{eqn:200to300}). Da
dem Fall 300 kein regelm"a"siges Intervall zugrundeliegt, m"ussen wir uns einzig
und allein mit der Folge $s_n$ behelfen. Um das Durchlaufen der Folge zu
vermeiden, verwenden wir einen Approximationsalgorithmus, mit dem wir uns an
$\vartheta$ ann"ahern.

Wir erkennen zun"achst, dass sich die Serienelemente aller $w$ Wochen
wiederholen. Wir errechnen erst einmal die Anzahl der Wochen $q$ vom Startdatum.
\begin{equation}
  q = \left[\frac{d_x - t'_s}{7 \textrm{ Tage}}\right]
\end{equation}
Das hei"st, $d_x$ liegt in der $q$. Woche seit $t'_s$. Nun beobachten wir
folgendes:
\begin{itemize}
\item Wenn $q \bmod w = 0$, dann liegt $d_x$ in einer Serienwoche, d. h. einer
  Woche, die an den Wochentagen in $\vec{\Theta}$ Termine der Serie enth"alt.
\item Wenn $q \bmod w \ne 0$, dann liegt $d_x$ in eine Zwischenwoche, d. h.
  einer Woche zwischen zwei Serienwochen.
\end{itemize}

\noindent Betrachten wir den ersten Fall ($q \bmod w = 0$). Der Wochentag $v$
des gegebenen Datums $d_x$ ermitteln wir mit
\begin{equation}v = \wdf(d_x)\end{equation}
Weiterhin sei $k$ die Anzahl der Elemente in $\vec{\Theta}$, also
\begin{equation}k = ||\vec{\Theta}||\end{equation}
Um die Differenz $\delta$ zum n"achsten Wochentag der Serie in Woche $q$ bzw.
der n"achstm"oglichen Woche zu ermitteln, verwenden wir folgenden Algorithmus:
\begin{equation}
  \delta = \left\{\begin{array}{ll}
    k = 1 : & 0 \\
    k > 1 : & \left\{\begin{array}{ll}
      v < \vec{\Theta}_k: & \vec{\Theta}_i - \vec{\Theta}_{1} \textrm{ mit }
      1 < i \le k \textrm{ und } \vec{\Theta}_{i-1} \le v < \vec{\Theta}_i \\
      v \ge \vec{\Theta}_k: & 7w \\
      \end{array}\right. \\
    \end{array}\right.
\end{equation}
(Bemerkung: Der Fall $k = 0$ ist nat"urlich nicht definiert und muss im Frontend
verhindert werden.)

\noindent Wenn der Wochentag $v$ nicht der letzte in der Wochentagsmenge
$\vec{\Theta}$ ist, ermitteln wir damit den n"achsten passenden Wochentag aus
der Menge, der nicht $v$ selbst ist. Falls es der letzte Wochentag in der Menge
ist, dann verwenden wir den ersten Wochentag aus $\vec{\Theta}$ in der
n"achstm"oglichen Serienwoche.

Nun ist der Errechnung von $\vartheta$ kein Problem mehr:
\begin{equation}
\label{eqn:theta300}\vartheta = t'_s + 7q + \delta
\end{equation}

\noindent Liegt $d_x$ nicht in einer Serienwoche (Fall $q \bmod w \ne 0$), dann
m"ussen wir lediglich die fehlenden Wochen bis zur n"achsten Serienwoche
addieren:
\begin{equation}\delta = 7w(\left[\frac{q}{w}\right] + 1) - 7q\end{equation}
und errechnen schlie"slich nach~\eqref{eqn:theta300} den Wert von $\vartheta$.


%--------------------------------------
%
% Faelle 400 und 500
%
%--------------------------------------
\subsection{F"alle 400 und 500}\label{ssec:approx400_500}
Hier l"asst sich f"ur $\varepsilon$ kein definitiver Wert festlegen, da die
Anzahl der Tage in einem Monat von 28 bis 31 variieren. Aus diesem Grund
kann~\eqref{eqn:nv} nicht angewendet werden und es muss eine Alternative gesucht
werden.

Wir berechnen zun"achst die Differenz $\Delta_m$ zwischen $d_x$ und $t'_s$ in
Monaten:
\begin{equation}\label{eqn:deltaM}
  \Delta_m = 12 (\yearf(d_x)-\yearf(t'_s)) - \monf(t'_s) + \monf(d_x)
\end{equation}
Hier ist jetzt eine Fallunterscheidung notwendig. Wir schaffen uns ein Pr"adikat
$P$, dass wahr ist, wenn $d_x$ in einem Monat liegt, in dem die Serie ein
Vorkommen hat. Es gilt also:
\begin{equation}P \equiv \Delta_m \bmod n = 0\end{equation}
Ist $P$ wahr, berechnen wir das Vorkommen $\vartheta'$ der Serie in diesem
Monat, d. h. im Monat $\monf(t'_s) + \Delta_m$ im Jahr $\yearf(t'_s)$. Im Fall
400 ist das:
\begin{equation}\label{eqn:thetaStrich400}
  \vartheta' = \calf(d_x, \monf(t'_s) + \Delta_m, \yearf(t'_s), , , )
\end{equation}
Im Fall 500 ist das:
\begin{equation}\label{eqn:thetaStrich500}
  \vartheta' = \calf(, \monf(t'_s) + \Delta_m, \yearf(t'_s), ws, , w)
\end{equation}
\begin{thm}
Ist $P$ wahr, d. h. liegt $d_x$ in einem Monat, in der die Serie ein Vorkommen
hat, muss dieses Vorkommen $\vartheta'$ sein.
\end{thm}
\begin{proof}[Beweis]
Wir beweisen zun"achst, dass $d_x$ und $\vartheta'$ im selben Monat und Jahr
liegt. Dazu schaffen wir uns eine Hilfsfunktion $m(d_x)$, die zu einem gegebenen
Datum die Anzahl der Monate errechnet:
\begin{equation}m(d_x) := 12\yearf(d_x) + \monf(d_x)\end{equation}
Nach~(\ref{eqn:deltaM}) gilt
\begin{eqnarray}
  \Delta_m & = & 12 (\yearf(d_x)-\yearf(t'_s)) - \monf(t'_s) + \monf(d_x) \\
    & = & 12\yearf(d_x) - 12\yearf(t'_s) + \monf(d_x) - \monf(t'_s) \\
    & = & 12\yearf(d_x) + \monf(d_x) - (12\yearf(t'_s) + \monf(t'_s)) \\
    & = & m(d_x) - m(t'_s)
\end{eqnarray}
F"ur Monat und Jahr von $\vartheta'$ gilt
\begin{eqnarray}
  m(\vartheta') & = & m(t'_s) + \Delta_m \\
    & = & m(t'_s) + m(d_x) - m(t'_s) \\
    & = & m(d_x)
\end{eqnarray}
Schlie"slich muss im Fall 400 f"ur den Tag des Vorkommens offensichtlich gelten:
\begin{equation}\dayf(\vartheta') = \dayf(t'_s)\end{equation}
Im Fall 500 gilt dagegen:
\begin{equation}
  \wdf(\vartheta') = \wdf(t'_s) \textrm{ und } \womf(\vartheta') = \womf(t'_s)
\end{equation}
Das ergibt sich nach~(\ref{eqn:thetaStrich400}) bzw.~(\ref{eqn:thetaStrich500}).
\end{proof}

Wenn $P$ wahr ist, kommt es darauf an, ob $d_x$ chronologisch vor oder hinter
dem Vorkommen $\vartheta'$ der Serie liegt. Gilt $d_x \le \vartheta'$, ist
offensichtlich $\vartheta'$ selbst das gesuchte Vorkommen. Gilt jedoch
$d_x > \vartheta'$, liegt das gesuchte Vorkommen $n$ Monate sp"ater, also zum
Zeitpunkt
\begin{equation*}\vartheta = \addff(\monf, \vartheta', n)\end{equation*}

Es bleibt noch der Fall $P$ ist falsch, d. h. $d_x$ liegt nicht in einem Monat,
in dem die Serie ein Vorkommen hat.
Dann ergibt sich $\vartheta$ wie folgt:
\begin{equation}\vartheta = \addff(\monf, \vartheta', n-\Delta_m)\end{equation}

\noindent Zusammenfassend gilt f"ur die F"alle 400 und 500 unter Zuhilfenahme
von $\vartheta'$ aus~(\ref{eqn:thetaStrich400}) und~(\ref{eqn:thetaStrich500}):
\begin{equation}\label{eqn:theta400}
  \vartheta = \left\{\begin{array}{ll}
  P : & \left\{\begin{array}{ll}
    d_x \le \vartheta' : & \vartheta' \\
    d_x > \vartheta' : & \addff(\monf, \vartheta', n) \\
    \end{array}\right. \\
  \overline{P} : & \addff(\monf, \vartheta', n-\Delta_m) \\
  \end{array}\right.
\end{equation}


%--------------------------------------
%
% Faelle 600 und 700
%
%--------------------------------------
\subsection{F"alle 600 und 700}
Wie schon in der Betrachtung der F"alle 400 und 500 (siehe
Seite~\pageref{ssec:approx400_500}) kann auch in diesen F"allen kein definitiver
Wert f"ur das Interval $\varepsilon$ angenommen werden, da die L"ange eines
Jahres zwischen 365 und 366 Tagen variiert. Auch die Verwendung der genauen
Dauer eines tropischen Erdjahres mit $\varepsilon \approx 365,242198$ Tage ist
f"ur kurze Zeitr"aume, die kein Schaltjahr enthalten, nicht geeignet.

Wir errechnen zun"achst das Datum $\vartheta'$ des Serienvorkommens im aktuellen
Jahr, d. h. im Jahr $\yearf(d_x)$. F"ur den Fall 600 gilt
\begin{equation}\vartheta' = \calf(d, m, \yearf(d_x), , , )\end{equation}
Im Fall 700 gilt
\begin{equation}\vartheta' = \calf(, m, \yearf(d_x), ws, , w)\end{equation}

Desweiteren repr"asentieren die Hilfswerte $y_s$ und $y_e$ den Anfang und das
Ende des aktuellen Jahres:
\begin{eqnarray}
  y_s & = & \calf(1, 1, \yearf(d_x), , , ) \\
  y_e & = & \calf(31, 12, \yearf(d_x), , , )
\end{eqnarray}

Liegt das gegebene Datum $d_x$ vor dem Serienvorkommen (oder ist es sogar das
Serienvorkommen), so ist $\vartheta'$ das gesuchte Datum. Andernfalls ist es das
Serienvorkommen im n"achsten Jahr. Zusammenfassend gilt also:
\begin{equation}
  \vartheta = \left\{\begin{array}{ll}
    y_s \le d_x \le \vartheta' : & \vartheta' \\
    \vartheta' < d_x \le y_e : & \addff(\yearf, \vartheta', 1) \\
    \end{array}\right.
\end{equation}



%================================================
%
%
% Weitere Formeln
%
%
%================================================
\section{Weitere Formeln}
Nachfolgend sind Formeln aufgelistet, die f"ur die L"osungsfindung bei oben
ausgef"uhrten Problemen von Nutzen sind.

Die Ermittlung der Differenz in Monaten $\Delta_m$ von zwei Daten $t_s$ und
$t_e$ (mit $t_s < t_e$) ergibt sich wie folgt:
\begin{equation}\label{eqn:monthDiff}
  \Delta_m = 12(\omega_y - \alpha_y) - \alpha_m + \omega_m
\end{equation}
Dabei bezeichnen $\alpha_m$ und $\omega_m$ die Monatsnummer und $\alpha_y$ und
$\omega_y$ die Jahreszahl des Anfangs- bzw. Enddatums $t_s$ und $t_e$.

Der Wochenbeginn $\alpha_w$ eines Datums $d$ berechnet sich nach folgender
Formel:
\begin{equation}\alpha_w(d) = d - (\wdf(d) - \eta + T) \mod T\end{equation}
Das ermittelte Datum $\alpha_w$ stellt dabei den ersten Tag in der Woche dar, in
der auch $d$ liegt. Dabei werden verschiedene Wochenbeginne ber"ucksichtigt.



%================================================
%
%
% Hinweise zur Implementation
%
%
%================================================
\section{Hinweise zur Implementation}
Es bietet sich an, ein darzustellendes Datum $d$ als einen ganzzahligen Wert in
Bezug auf einen Urwert (z. B. dem 1. Januar 1970) anzusehen. F"ur die
Implementation in Java bietet sich die Verwendung der Klasse
\texttt{java.util.Calendar} an, die umfangreiche M"oglichkeiten zur Berechnung
offeriert.

Eine ganze Reihe von wichtigen Datumsfunktionen wurde in die Klasse
\texttt{org.amcworld.util.date.DateTimeUtils} eingebaut. In sYnergy werden die
Serienberechnungen in den drei Klassen \texttt{CalendarEntry},
\texttt{CalendarSeries} und \texttt{CalendarSeriesField}, alle aus dem Package
\texttt{org.amcworld.synergy.cms.\-data}.
\end{document}
