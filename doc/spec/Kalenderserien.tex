%
% Kalenderserien.tex
%
% $Id: Kalenderserien.tex 474 2006-06-07 17:15:29Z danny $
%
% Copyright (c) 1998-2011 AMC World Technologies GmbH
% Fischerinsel 1, D-10179 Berlin, Deutschland
% All Rights Reserved.
%
% This software is the confidential and proprietary information of AMC World
% Technologies GmbH ("Confidential Information"). You shall not disclose such
% Confidential Information and shall use it only in accordance with the terms
% of the licence agreement you entered into with AMC World Technologies GmbH.
%


%================================================
%
%
% Praeambel
%
%
%================================================
\documentclass[a4paper]{article}

%--------------------------------------
%
% Packages
%
%--------------------------------------
\usepackage{amsmath}
\usepackage{amssymb}
\usepackage{amsthm}
\usepackage{ngerman}
\usepackage{tabularx}


%--------------------------------------
%
% Kommandos
%
%--------------------------------------
\newcommand*{\dayf}{\mathrm{day}}
\newcommand*{\monf}{\mathrm{mon}}
\newcommand*{\yearf}{\mathrm{year}}
\newcommand*{\wdf}{\mathrm{wd}}
\newcommand*{\weekf}{\mathrm{week}}
\newcommand*{\womf}{\mathrm{wom}}
\newcommand*{\linf}{\mathrm{lin}}
\newcommand*{\calf}{\mathrm{cal}}
\newcommand*{\sgn}{\mathrm{sgn}}
\newcommand*{\datev}[1]{\texttt{`#1'}}
\newcommand*{\gff}{\mathrm{gf}}
\newcommand*{\sff}{\mathrm{sf}}
\newcommand*{\addff}{\boxplus}
\newcommand*{\pot}[1]{\mathcal{P}(#1)}
\newcommand*{\mondays}{\mathfrak{D}}
\newcommand*{\months}{\mathfrak{M}}
\newcommand*{\wkdays}{\mathfrak{W}}
\newcommand*{\years}{\mathfrak{Y}}
\newcommand*{\weeksyr}{\mathfrak{W}_Y}
\newcommand*{\weeksmo}{\mathfrak{W}_M}
\renewcommand{\theequation}{\thesection.\arabic{equation}}
\numberwithin{equation}{section}
\newtheorem{dfn}{Definition}
\newtheorem{thm}{Satz}
\newtheorem{lmm}{Lemma}


%--------------------------------------
%
% Laengen
%
%--------------------------------------
\newlength{\widthTab}  % Breite von bestimmten Tabellen


%--------------------------------------
%
% Titel
%
%--------------------------------------
\title{Berechnung von Kalenderserien}
\author{Daniel Ellermann}
\date{5. Dezember 2011}


%--------------------------------------
%
% Dokumentstart
%
%--------------------------------------
\begin{document}
\maketitle
\tableofcontents



%================================================
%
%
% Inspiration
%
%
%================================================
\section{Betrifft}
Die Berechnung und Verwaltung von Kalenderserien ist eine z. T. nicht triviale
Aufgabe. SpringCRM unterst"utzt die wichtigsten Muster f"ur Terminserien, ohne
dabei den Benutzer unn"otig zu verwirren. Es bestand der Anspruch, f"ur
s"amtliche zur Verf"ugung stehenden Serienmuster eine Formel zur Berechnung
des Zeitpunkts bei einer gegebenen Anzahl $n$ von Terminen der Serie zu haben.

Weiterhin bietet dieser Artikel wertvolle Formeln, die f"ur die
Kalenderberechnung von ausschlaggebender Bedeutung sind.



%================================================
%
%
% Serienmuster
%
%
%================================================
\section{Serienmuster}
Die definierten Serientypen in SpringCRM finden Sie in Tabelle
\ref{tab:pattern} auf Seite \pageref{tab:pattern}. Jedes Muster hat die Form

\vspace{8pt}
\texttt{fall d m y wd [ ursprung ]}
\vspace{8pt}

\noindent Dabei stellt \texttt{fall} die Fallnummer nach
Tabelle~\ref{tab:pattern}, Spalte "`Fall"' dar, sowie \texttt{d} das
Tagesmuster, \texttt{m} das Monatsmuster, \texttt{y} das Jahresmuster und
\texttt{wd} das Wochentagsmuster.

Jedes dieser Muster besteht aus einem Basiswert und einem optionalen
Qualifizierer. F"ur jedes der vier Muster (Tagesmuster, Monatsmuster,
Jahresmuster und Wochentagsmuster) muss genau einer der folgenden m"oglichen
Basiswerte angegeben werden. Dabei sei $v$ ein gegebener Wert aus dem
Wertebereich des jeweiligen Feldes. Zu jedem m"oglichen Fall geben wir ein
Pr"adikat $P$ an, das aussagt ob $v$ zu dem Wert des Feldes passt.
\begin{itemize}
\item \texttt{*} (beliebiger Wert). Das Feld passt auf jeden beliebigen Wert. Es
  gilt $P \equiv \top$.
\item \texttt{\textit{x}} (einfacher Wert). Das Feld passt dann und nur dann,
  wenn es den Wert \texttt{\textit{x}} hat, d. h. $P \equiv v = x$.
\item \texttt{\textit{a}-\textit{b}} (Bereichswert). Repr"asentiert Werte im
  Bereich von \texttt{\textit{a}} bis \texttt{\textit{b}} (beide
  eingeschlossen). Es gilt $P \equiv v \in [a;\,b]$.
\item \texttt{\textit{a}, \textit{b}, \ldots} (Aufz"ahlungswert). Repr"asentiert
  die Werte \texttt{\textit{a}}, \texttt{\textit{b}} usw. Dabei gilt
  $P \equiv v \in M$ mit $M = \{a,\,b,\dots\}$.
\item \texttt{+\textit{x}} bzw. \texttt{-\textit{x}} (Relativwert). Stellt einen
  relativen Wert zu einem gegebenen Ursprung $d_u = $ \texttt{\textit{ursprung}}
  dar, der am Ende des Musters angegeben wird. Wird kein Ursprung spezifiziert,
  wird der 1. Januar angenommen. Beispiel: Das Muster \texttt{+49 * * * E} mit
  \texttt{E} als Kennzeichnung des Ostersonntags (Eastern) als Ursprung
  repr"asentiert den Pfingstsonntag, der 49 Tage nach dem Ostersonntag liegt.
  Relativwerte sind nur im Tages- und Monatsmuster erlaubt. Es gilt
  $P \equiv v = d_u + x$.
\end{itemize}

\noindent Dar"uber hinaus kann jeder dieser Basiswerte mit den folgenden
optionalen Qualifizierern gekennzeichnet werden. In den nachfolgenden
Definitionen steht \texttt{\textit{b}} f"ur einen Basiswert.
\begin{itemize}
\item \texttt{\textit{b}/\textit{n}} (Intervallwert). Das Feld passt auf jedes
  \texttt{\textit{n}}. Vorkommen des Wertes \texttt{\textit{b}}. Dabei darf
  \texttt{\textit{b}} allerdings kein Relativwert sein. Beispiel: der Wert
  \texttt{*/2} im Monatsfeld bedeutet aller zwei Monate; der Wert \texttt{2,4/3}
  im Wochentagsfeld bedeutet Montag und Mittwoch aller drei Wochen.
\item \texttt{\textit{b}:\textit{y}} (Ordinalwert). Bedeutet den
  \texttt{\textit{y}}. Wert \texttt{\textit{b}}. Ordinalwerte sind nur im
  Wochentagsfeld erlaubt. M"oglich sind auch negative Werte f"ur
  \texttt{\textit{y}}, wobei diese dann vom Ende eines Monats gerechnet werden.
  Beispiel: der Wert \texttt{5:2} stellt den zweiten Donnerstag dar, der Wert
  \texttt{3:-1} bedeutet den letzten Dienstag im Monat. Momentan werden bei
  Ordinalwerten als zugeh"origer Basiswert nur \texttt{*} und ein exakter Wert
  unterst"utzt.
\end{itemize}

\setcounter{footnote}{1}
\begin{table}[ht]
  \caption{Muster f"ur Terminserien}
  \label{tab:pattern}
  \begin{tabularx}{\textwidth}{rccccXc}
    \hline
    Fall & \multicolumn{4}{c}{Muster} & Beschreibung & Intervall $\varepsilon$ \\
    & Tag & Mon & Jahr & WT & & \\
    \hline
    0 & \multicolumn{4}{c}{-} & keine Terminserie & - \\
    10 & \texttt{*/$n$} & \texttt{*} & \texttt{*} & \texttt{*} &
      aller $n$ Tage & $n$ Tage \\
    20 & \texttt{*} & \texttt{*} & \texttt{*} & \texttt{1-5} &
      jeden Arbeitstag & - \\
    30 & \texttt{*} & \texttt{*} & \texttt{*} & \texttt{$ws$/$n$} &
      an den Wochentagen $ws$ aller $n$ Wochen & - \\
    40 & \texttt{$d$} & \texttt{*/$n$} & \texttt{*} & \texttt{*} &
      am $d$. Tag jedes $n$. Monats & $n$ Monate \\
    50 & \texttt{*} & \texttt{*/$n$} & \texttt{*} & \texttt{$ws$:$w$} &
      am $w$. Wochentag $ws$ jedes $n$. Monats &
      $n$ Monate\footnotemark[\ref{ft:oneMonth}] \\
    60 & \texttt{$d$} & \texttt{$m$} & \texttt{*/1} & \texttt{*} &
      am $d$. des Monats $m$ (Geburtstagsserie) & 1 Jahr \\
    70 & \texttt{*} & \texttt{$m$} & \texttt{*/1} & \texttt{$ws$:$w$} &
      am $w$. Wochentag $ws$ im Monat $m$ (Muttertagsserie) &
      1 Jahr\footnotemark[\ref{ft:oneMonth}] \\
    80 & \multicolumn{4}{c}{-} & Sonderf"alle, z. B. Feiertage & - \\
    \hline
  \end{tabularx}
\end{table}
\footnotetext{\label{ft:oneMonth}Im Monat des Vorkommens des Serienelements ist
  der $w$. Wochentag zu w"ahlen.}

\label{badInterval}
Bei den Intervallen $n$ Monat(e) bzw. 1 Jahr ist zu beachten, dass sich diese
Werte programmiertechnisch schlecht abbilden lassen. Eine Umrechnung in einen
Millisekundenwert ist nicht m"oglich, da die L"ange eines Monats bzw. eines
Jahres variieren kann.



%================================================
%
%
% Voraussetzungen
%
%
%================================================
\section{Voraussetzungen}
In den nachfolgenden Berechnungen seien folgende Definitionen gegeben.


%--------------------------------------
%
% Wochendaten
%
%--------------------------------------
\subsection{Wochendaten}
F"ur Daten die Woche betreffend, definieren wir folgendes:
\begin{dfn}[Wochenl"ange]
Die Anzahl der Tage in einer Woche des gregorianischen Kalenders bezeichnen wir
mit $T$. Es gilt:
\begin{equation}T := 7\end{equation}
\end{dfn}
\noindent In anderen Regionen und Kalendarien kann $T$ auch einen anderen Wert
annehmen.

\begin{dfn}[Wochentage]\label{def:weekdays}
  Die Menge aller Wochentage des gregorianischen Kalenders wird mit $\wkdays$
  bezeichnet:
  \begin{equation}
    \wkdays := \{ d_w : 0 < d_w \le T\},\ \wkdays \subset \mathbb{N}
  \end{equation}
  Die Wochentage werden durch Indizes beginnend mit eins
  wie in Tabelle~\ref{tab:weekDays} definiert
  beschrieben.\footnotemark[\ref{ft:weekdays}]
  \par
  \begin{table}[ht]
  \caption{Wochentage $\wkdays$}\label{tab:weekDays}
  \begin{tabularx}{\textwidth}{X|ccccccc}
    \hline
    Wochentag & Montag & Dienstag & Mittwoch & Donnerstag \\
    Index $d_w \in \wkdays$ & 2 & 3 & 4 & 5 \\
    \hline
    Wochentag & Freitag & Samstag & Sonntag & \\
    Index $d_w \in \wkdays$ & 6 & 7 & 1 & \\
    \hline
  \end{tabularx}\end{table}
\end{dfn}
\footnotetext{\label{ft:weekdays}Die Indizes der Wochentage entspricht der
Nummerierung der Wochentage in der Java-Klasse \texttt{java.util.Calendar}.}

\begin{dfn}[Wochenanfang]
  Der Index des Wochentages des ersten Tags in der Woche wird mit $\eta$
  bezeichnet und es gilt:
  \begin{equation}\eta \in \wkdays\end{equation}
  In Deutschland gilt $\eta := 2$, in den USA gilt $\eta := 1$.
\end{dfn}


%--------------------------------------
%
% Tage, Wochen, Monate und Jahre
%
%--------------------------------------
\subsection{Tage, Wochen, Monate und Jahre}
\begin{dfn}[Maximale Anzahl der Monate im Jahr]
  Die maximale Anzahl der Monate in einem Jahr des gregorianischen Kalenders
  bezeichnen wir mit $M_{max}$. Es gilt:
  \begin{equation}M_{max} := 12\end{equation}
\end{dfn}
\noindent In anderen Regionen und Kalendarien kann $M_{max}$ auch einen anderen
Wert annehmen.
\begin{dfn}[Monate]
  Die Menge aller Monate wird mit $\months$ bezeichnet und es gilt:
  \begin{equation}\months := \{ m : 0 \le m < M_{max}\}\end{equation}
  Die Monate werden durch Indizes beginnend mit null wie in
  Tabelle~\ref{tab:months} definiert beschrieben.\footnotemark[\ref{ft:months}]
  \par
  \begin{table}[ht]
  \caption{Monate $\months$}\label{tab:months}
  \begin{tabularx}{\textwidth}{X|ccccccc}
    \hline
    Monat & Januar & Februar & M"arz & April \\
    Index $m \in \months$ & 0 & 1 & 2 & 3 \\
    Anzahl Tage $D_m$ & 31 & 28 oder 29 & 31 & 30 \\
    \hline
    Monat & Mai & Juni & Juli & August \\
    Index $m \in \months$ & 4 & 5 & 6 & 7 \\
    Anzahl Tage $D_m$ & 31 & 30 & 31 & 31 \\
    \hline
    Monat & September & Oktober & November & Dezember \\
    Index $m \in \months$ & 8 & 9 & 10 & 11 \\
    Anzahl Tage $D_m$ & 30 & 31 & 30 & 31 \\
    \hline
  \end{tabularx}\end{table}
\end{dfn}
\footnotetext{\label{ft:months}Die Indizes der Monate entspricht der
Nummerierung der Monate in der Java-Klasse \texttt{java.util.Calendar}. Deshalb
erfolgt die Nummerierung un"ublicherweise nullbasiert.}

\begin{dfn}[Maximale Anzahl der Tage jedem Monat]
  Die maximale Anzahl der Tage in einem beliebigen Monat im gregorianischen
  Kalender bezeichnen wir mit
  $D_{max}$. Es gilt:
  \begin{equation}D_{max} := 31\end{equation}
\end{dfn}
\noindent In anderen Regionen und Kalendarien kann $D_{max}$ auch einen anderen
Wert annehmen.
\begin{dfn}[Maximale Anzahl der Tage in einem bestimmten Monat]
  Die maximale Anzahl der Tage in einem bestimmten Monat $m \in \months$ im
  gregorianischen Kalender ohne Ber"ucksichtigung der Schaltjahre bezeichnen wir
  mit $D_m$ und ist in Tabelle~\ref{tab:months} definiert. Es gilt:
  \begin{equation}D_m \le D_{max}\end{equation}
\end{dfn}
\begin{dfn}[Tage im Monat]
  Die Menge aller Tage in einem beliebigen Monat $m \in \months$ wird mit
  $\mondays$ bezeichnet und es gilt:
  \begin{equation}\mondays := \{ d : 0 < d \le D_{max}\}\end{equation}
  Die Menge aller Tage in einem Monat $m \in \months$ wird mit $\mondays_m$
  bezeichnet und es gilt:
  \begin{equation}
    \mondays_m := \{ d : 0 < d \le D_m\} \subseteq \mondays
  \end{equation}
\end{dfn}
\begin{dfn}
  Die Menge der Jahre $\years$ ist wie folgt definiert:
  \begin{equation}\years := \mathbb{Z} \backslash 0\end{equation}
\end{dfn}

\noindent Weiterhin seien noch folgende Mengen definiert:
\begin{dfn}
  Die Menge der Wochen $\weeksyr$ in einem Jahr des gregorianischen Kalenders
  ist wie folgt definiert:
  \begin{equation}
    \weeksyr := \{ w : 1 \le w \le 53 \},\ \weeksyr \subset \mathbb{N}
  \end{equation}
\end{dfn}
\begin{dfn}
  Die Menge der Wochen $\weeksmo$ in einem Monat $m \in \months$ ist wie folgt
  definiert:
  \begin{equation}
    \weeksmo := \{ w : 1 \le w \le 5 \},\ \weeksmo \subset \mathbb{N}
  \end{equation}
\end{dfn}


%--------------------------------------
%
% Terminserien
%
%--------------------------------------
\subsection{Terminserien}
F"ur die betrachteten Terminserien definieren wir folgendes:

\begin{dfn}[Anzahl der Elemente einer Terminserie]\label{def:numOfElements}
  Die vorgegebene Anzahl von Terminen innerhalb einer Terminserie bezeichnen wir
  mit $n$. Es gilt:
  \begin{equation}n \in \mathbb{N},\ n > 0\end{equation}
\end{dfn}

\begin{dfn}[Anfangs- und Endpunkt einer Terminserie]
  Die beiden benutzerdefinierten Anfangs- und Endzeitpunkte der Serie werden mit
  $t_s$ bzw. $t_e$ mit $t_s,\,t_e \in \Psi$ bezeichnet. Diese Zeitpunkte selbst
  m"ussen nicht zur Serie geh"oren.
\end{dfn}

\begin{dfn}[Kalibrierter Startpunkt]
  Der kalibrierte Startpunkt einer Terminserie wird mit $t'_s \in \Psi$
  bezeichnet. Dies kann entweder das Startdatum der Terminserie sein, oder es
  handelt sich dabei um einen Betrachtungszeitpunkt. In jedem Fall muss $t'_s$
  selbst ein Element der Serie sein.
\end{dfn}

\begin{dfn}[Terminserie als Folge]\label{def:seriesAsSeries}
  Eine Terminserie stellt eine Folge $s_n$ von Serienelementen dar, f"ur die
  gilt:
  \begin{eqnarray}
    s_1 & := & t'_s \\
    s_{i+1} & := & s_i + \varepsilon
  \end{eqnarray}
\end{dfn}

\noindent Manchmal ben"otigen wir auch die Serienelemente in der
Mengenschreibweise und beschreiben das durch folgende Definition:
\begin{dfn}[Serienelementmenge]
  Die geordnete Menge aller Elemente einer Serie bezeichnen wir als $\vec{S}$
  und es gilt:
  \begin{equation}
    \vec{S} := \forall i : 0 < i \le n :\ \{ x : x = s_i \}
  \end{equation}
\end{dfn}


%--------------------------------------
%
% Funktionen
%
%--------------------------------------
\subsection{Funktionen}
Um Zusammenhang mit dem landesspezifisch unterschiedlichen Wochenbeginn $\eta$
ergibt sich die Notwendigkeit, die Nummern Wochentage $w \in \wkdays$ zu
linearisieren. Dabei werden die Nummern so angeordnet, dass sie den Ablauf einer
Woche repr"asentieren.
\begin{dfn}[Linearisierungsfunktion]\label{def:lin}
  Die Linearisierungsfunktion $\linf$ ist f"ur alle $w \in \wkdays$ wie folgt
  definiert:
  \begin{equation}
  \begin{split}
    & \linf :\ \wkdays \mapsto [\eta;\,\eta + T[ \\
    & \linf(w) := w + \left\{\begin{array}{ll}
        w < \eta : & T \\
        w \ge \eta : & 0 \\
      \end{array}\right. =
      \ w + T\left\lfloor\frac{1}{2}(\sgn(\eta - w) + 1)\right\rfloor
  \end{split}
  \end{equation}
\end{dfn}

\noindent Beispiel: Bei $\eta = 1$ ergeben sich f"ur die Wochentage Sonntag bis
Samstag beispielsweise die Nummern 1 bis 7. Bei $\eta = 2$ dagegen f"ur die
Wochentage Montag bis Sonntag die Nummern 2 bis 8. Schlie"slich sind die
Wochentage Samstag bis Freitag bei $\eta = 7$ mit 7 bis 13 nummeriert.

\begin{thm}[Maximum der Linearisierungsfunktion]\label{thm:maxLin}
  Das Maximum der Linearisierungsfunktion ist:
  \begin{equation}\forall w,\,\eta \in \wkdays : \linf(w) < 2T\end{equation}
\end{thm}
\begin{proof}[Beweis]
  Der Wertebereich von $\linf$ ist $[\eta;\,\eta + T[$. Das hei"st,
  $\linf(w) < \eta + T$. F"ur $\eta$ gilt $0 < \eta \le T$. Das Maximum von
  $\eta$ ist also $T$. Dann gilt $\linf(w) < T + T = 2T$.
\end{proof}

F"ur den Beweis der nachfolgenden Umkehrfunktion ben"otigen wir folgenden
Hilfssatz:
\begin{lmm}[Modulo-Operation im eingeschr"ankten Bereich]\label{lmm:modulo}
  F"ur Modulo Operation im Bereich $D = [0;\,2n[\ \subseteq \mathbb{N}$ mit
  $n \in \mathbb{N}$ und $v \in D$ gilt:
  \begin{equation}
    v \bmod n = \left\{\begin{array}{ll}
        0 \le v < n & : v \\
        n \le v < 2n & : v - n \\
      \end{array}\right.
  \end{equation}
\end{lmm}
\begin{proof}[Beweis]
  Zun"achst sei $t = \lfloor\frac{v}{n}\rfloor$. Nach Definition der
  Modulo-Operation gilt:
  \begin{equation}s = v \bmod n \Longleftrightarrow v = tn + s\end{equation}

  \noindent Wir zeigen zuerst den Fall $0 \le v < n$. Sei Dann gilt $t = 0$ so
  dass auch $v = 0 + s = v$ gilt. Im zweiten Fall $n \le v < 2n$ gilt $t = 1$
  und damit $v = n + s = n + v - n = v$.
\end{proof}

\noindent F"ur Umkehrfunktion der Linearisierungsfunktion $\linf^{-1}$ gilt
folgendes:
\begin{thm}[Umkehrfunktion der Linearisierungsfunktion]
  \begin{equation}
  \begin{split}
    & \linf^{-1} :\ [\eta;\,\eta + T[\ \mapsto \wkdays \\
    & \linf^{-1}(x) := (x - 1) \bmod T + 1
  \end{split}
  \end{equation}
\end{thm}
\begin{proof}[Beweis]
  Wir zeigen zun"achst $\linf(\linf^{-1}(x)) = x$:
  \begin{equation}
  \begin{split}
    x & = \linf(\linf^{-1}(x)) \\
    & = \linf((x - 1) \bmod T + 1) \\
    & = (x - 1) \bmod T + 1 + T\left\lfloor\frac{1}{2}
      (\sgn(\eta - ((x - 1) \bmod T + 1)) + 1)\right\rfloor \\
  \end{split}
  \end{equation}
  Hier unterscheiden wir drei F"alle.

  \noindent\textbf{Fall I $(x - 1) \bmod T + 1 < \eta$:} Dann ist
  $\sgn(\eta - ((x - 1) \bmod T + 1)) = 1$. Dann gilt weiter:
  \begin{equation}
  \begin{split}
    x & = (x - 1) \bmod T + 1 + T\left\lfloor\frac{1}{2} (1 + 1)\right\rfloor \\
    & = (x - 1) \bmod T + 1 + T \\
  \end{split}
  \end{equation}
  Nach Definition~\ref{def:lin} ist in betrachteten Fall $x \ge T$ und nach
  Lemma~\ref{lmm:modulo} gilt:
  \begin{equation}
  \begin{split}
    x & = x - 1 - T + 1 + T \\
    & = x
  \end{split}
  \end{equation}

  \noindent\textbf{Fall II $(x - 1) \bmod T + 1 = \eta$:} Dann ist
  $\sgn(\eta - ((x - 1) \bmod T + 1)) = 0$. Dann gilt weiter:
  \begin{equation}
  \begin{split}
    x & = (x - 1) \bmod T + 1 + T\left\lfloor\frac{1}{2} (0 + 1)\right\rfloor \\
    & = (x - 1) \bmod T + 1 \\
  \end{split}
  \end{equation}

  \noindent\textbf{Fall III $(x - 1) \bmod T + 1 \ge \eta$:} Dann ist
  $\sgn(\eta - ((x - 1) \bmod T + 1)) = -1$. Dann gilt weiter:
  \begin{equation}
  \begin{split}
    x & = (x - 1) \bmod T + 1 + T\left\lfloor\frac{1}{2} (1 - 1)\right\rfloor \\
    & = (x - 1) \bmod T + 1 \\
  \end{split}
  \end{equation}

  \noindent In den F"allen II und III ist nach Definition~\ref{def:lin} $x < T$
  und nach Lemma~\ref{lmm:modulo} gilt:
  \begin{equation}
  \begin{split}
    x & = x - 1 + 1 \\
    & = x
  \end{split}
  \end{equation}

  \noindent Nun beweisen wir $\linf^{-1}(\linf(w)) = w$:
  \begin{equation}
  \begin{split}
    w & = \linf^{-1}(\linf(w)) \\
    & = \linf^{-1}(w + T\left\lfloor\frac{1}{2}
      (\sgn(\eta - w) + 1)\right\rfloor) \\
    & = (w + T\left\lfloor\frac{1}{2}
      (\sgn(\eta - w) + 1)\right\rfloor - 1) \bmod T + 1 \\
  \end{split}
  \end{equation}
  Hier unterscheiden wir zwei F"alle.

  \noindent\textbf{Fall I $w < \eta$:} Dann ist $\sgn(\eta - w) = 1$. Dann gilt
  weiter:
  \begin{equation}w = (w + T - 1) \bmod T + 1\end{equation}
  Da $w + T - 1 \ge T$ ist gilt nach Lemma~\ref{lmm:modulo}:
  \begin{equation}
  \begin{split}
    w & = w + T - 1 - T + 1 \\
    & = w
  \end{split}
  \end{equation}

  \noindent\textbf{Fall II $w \ge \eta$:} Dann ist entweder $\sgn(\eta - w) = 0$
  oder $\sgn(\eta - w) = -1$. Dann gilt in beiden F"allen weiter:
  \begin{equation}w = (w - 1) \bmod T + 1\end{equation}
  Da $w - 1 < T$ ist gilt nach Lemma~\ref{lmm:modulo}:
  \begin{equation}
  \begin{split}
    w & = w - 1 + 1 \\
    & = w
  \end{split}
  \end{equation}
\end{proof}

\noindent Datumswerte wie der 13.05.1975 oder der 26.10.2011 werden durch die
Menge $\Psi$ beschrieben:
\begin{dfn}[Menge der Datumswerte]
  Die Menge der Datumswerte bestehend aus einem Tag $d \in \mondays$, einem
  Monat $m \in \months$ und einem Jahr $y \in \years$ wird mit $\Psi$
  bezeichnet.
\end{dfn}

Es seien eine Reihe von Funktionen gegeben, die bestimmte Elemente eines
gegebenen Datums $d$ extrahieren.
\begin{dfn}[Datumsfunktionen]
  Zur Extrahierung eines bestimmten Kalendarelements eines gegebenen Datums
  $d \in \Psi$ seien die Funktionen in Tabelle~\ref{tab:dateFunctions}
  definiert.
  \begin{table}[tp]
  \caption{Datumsfunktionen}\label{tab:dateFunctions}
  \begin{tabularx}{\textwidth}{lXl}
  \hline
  Funktion & Beschreibung & Beispiel \\
  \hline
  $\dayf : \Psi \mapsto \mondays$ & berechnet den Tag von $d$. &
    $\dayf(d) = 13$ \\
  $\monf : \Psi \mapsto \months$ & berechnet den Monat von $d$. &
    $\monf(d) = 5$ \\
  $\yearf : \Psi \mapsto \years$ & berechnet das Jahr von $d$. &
    $\yearf(d) = 2002$ \\
  $\wdf : \Psi \mapsto \wkdays$ & berechnet den Wochentagsindex von $d$. &
    $\wdf(d) = 1$ \\
  $\weekf : \Psi \mapsto \weeksyr$ & berechnet die Wochennummer von $d$. &
    $\weekf(d) = 20$ \\
  $\womf : \Psi \mapsto \weeksmo$ & berechnet die Wochennummer von $d$
    innerhalb eines Monats. & $\womf(d) = 3$ \\
  \hline
  \multicolumn{3}{c}{Die Beispiele setzen $d = \datev{13.05.2002}$ voraus.}
  \end{tabularx}\end{table}
\end{dfn}

\noindent Weiterhin brauchen wir eine Funktion, die aus angegebenen
Kalenderelementen das zugeh"orige Datum errechnet:
\begin{dfn}[Kalenderfunktion]
  Die Funktion $\calf$ berechnet aus den gegebenen, optionalen Elementen Tag im
  Monat, Monat, Jahr, Wochentag, Wochennummer und Wochennummer innerhalb eines
  Monats das zugeh"orige Datum. Dabei k"onnen Elemente, die nicht von Bedeutung
  sind, weggelassen werden, wobei die Anzahl der Kommata unver"anderlich bleiben
  muss. Es gilt:
  \begin{equation}
  \begin{split}
    & \calf :\ \mondays,\,\months,\,\years,\,\weeksyr,\,\weeksmo \mapsto \Psi \\
    & q = \calf(d,\,m,\,y,\,wd,\,week,\,wom) \Leftrightarrow \\
    &   \qquad\qquad(\exists\ d \Rightarrow \dayf(q) = d)\ \wedge \\
    &   \qquad\qquad(\exists\ m \Rightarrow \monf(q) = m)\ \wedge \\
    &   \qquad\qquad(\exists\ y \Rightarrow \yearf(q) = y)\ \wedge \\
    &   \qquad\qquad(\exists\ wd \Rightarrow \wdf(q) = wd)\ \wedge \\
    &   \qquad\qquad(\exists\ week \Rightarrow \weekf(q) = week)\ \wedge \\
    &   \qquad\qquad(\exists\ wom \Rightarrow \womf(q) = wom) \\
  \end{split}
  \end{equation}
\end{dfn}
\noindent Beispiel: $\calf(13,\,5,\,2002,\,,\,,) = \datev{13.05.2002}$.

Zum allgemeinen Arbeiten mit Kalenderfeldern wollen wir noch zwei Funktionen
definieren, die den Wert eines Kalenderfeldes liefern und setzen, sowie eine
Funktion zum Ver"andern eines Kalenderfeldes.
\begin{dfn}[lesender Kalenderfeldzugriff]
  Die Funktion $\gff$ (f"ur \emph{get calendar field}) ermittelt den Wert eines
  Kalenderfeldes $f$ in einem Datum $d$. Sie ist wie folgt definiert:
  \begin{equation}
  \begin{split}
    & \gff :\ \{\dayf,\,\monf,\,\yearf,\,\wdf,\,\weekf,\,\womf\}
      \mapsto \mathbb{N} \\
    & \gff(f,\,d) := f(d)
  \end{split}
  \end{equation}
\end{dfn}
\noindent Beispiel: $\gff(\monf,\,\datev{13.05.2002}) = 5$.

\begin{dfn}[schreibender Kalenderfeldzugriff]
  Die Funktion $\sff$ (f"ur \emph{set calendar field}) setzt den Wert eines
  Kalenderfeldes in einem Datum. Sie ist wie folgt definiert:
  \begin{equation}
  \begin{split}
    & \sff :\ \{\dayf,\,\monf,\,\yearf,\,\wdf,\,\weekf,\,\womf\},\,\Psi,\,
      \mathbb{N} \mapsto \Psi \\
    & \sff(f,\,d,\,q) := \left\{\begin{array}{ll}
        f = \dayf : & \calf(q,\,\monf(d),\,\yearf(d),\,,\,,\,) \\
        f = \monf : & \calf(\dayf(d),\,q,\,\yearf(d),\,,\,,\,) \\
        f = \yearf : & \calf(\dayf(d),\,\monf(d),\,q,\,,\,,) \\
        f = \wdf : & \calf(,\,\monf(d),\,\yearf(d),\,q,\,\weekf(d),\,) \\
        f = \weekf : & \calf(,\,\monf(d),\,\yearf(d),\,\wdf(d),\,q,\,) \\
        f = \womf : & \calf(,\,\monf(d),\,\yearf(d),\,\wdf(d),\,,\,q) \\
      \end{array}\right.
  \end{split}
  \end{equation}
\end{dfn}
\noindent Beispiel: $\sff(\monf,\,\datev{13.05.2002},\,8) = \datev{13.08.2002}$.

\begin{dfn}["andernder Kalenderfeldzugriff]
  Die Funktion $\addff$ "andert den Wert eines Kalenderfeldes in einem Datum um
  einen gegebenen Wert. Sie ist wie folgt definiert:
  \begin{equation}
  \begin{split}
    & \addff :\ \{\dayf,\,\monf,\,\yearf,\,\wdf,\,\weekf,\,\womf\},\,\Psi,\,
      \mathbb{N} \mapsto \Psi \\
    & \addff(f,\,d,\,q) := \sff(f,\,d,\,\gff(f,\,d) + q)
  \end{split}
  \end{equation}
\end{dfn}

\noindent Die angegebenen Wochentage $ws \in \wkdays$ in einem Serienmuster
k"onnen wir als eine geordnete Liste auf\mbox{}fassen, die den Index jedes
Wochentags aus $ws$ enth"alt. Dazu verwenden wir die folgenden Definitionen:
\begin{dfn}[geordnete Liste]
  Eine geordnete Liste von beliebigen Werten $a_1,\,a_2,\,\dots,\,a_n$ wird in
  der Form $\vec{L} = \langle a_1,\,a_2,\,\dots,\,a_n\rangle$ notiert. Die Menge
  aller Listen $\vec{N} = \langle a_1,\,a_2,\,\dots,\,a_n\rangle$ mit
  $a_i \in \mathbb{N}$ f"ur alle $1 \le i \le n$ wird mit $\vec{\mathbb{N}}$
  bezeichnet.
\end{dfn}
\begin{dfn}[geordnete Liste von Wochentagen]
  Eine geordnete Liste von Wochentagen $w_1,\,w_2,\,\ldots,\,w_n \in \wkdays$
  wird mit $\vec{W}$ bezeichnet. Es gilt
  \begin{equation}
    \vec{W} := \langle w_1,\,w_2,\,\dots,\,w_n\rangle\textrm{ mit }
    w_1 < w_i < w_{i+1} \textrm{ f"ur alle } 1 < i < n
  \end{equation}
\end{dfn}

\noindent F"ur Bereichswerte gilt folgendes:
\begin{thm}\label{thm:range}
  Ein Bereichswert \texttt{a-b} kann wie folgt als geordnete Liste dargestellt
  werden:
  \begin{equation}
    \texttt{a-b} = [a;\,b] \iff \langle a,\,a+1,\,\dots,\,b\rangle
  \end{equation}
\end{thm}
\begin{proof}[Beweis]Der Beweis ist trivial.\end{proof}

\noindent Wir ben"otigen eine Funktion, die den Index eines Wochentages in einer
geordneten Liste von Wochentagen sucht. Folgende Definition soll das
liefern:
\begin{dfn}
  \begin{equation}
  \begin{split}
    & \pi :\ \vec{\mathbb{N}},\,\wkdays \mapsto \mathbb{N} \\
    & \pi(\vec{M},\,w) := \left\{\begin{array}{ll}
    \exists\ q_i \in \vec{M} \textrm{ mit } q_i = w : & i \\
    \nexists\ q_i \in \vec{M} \textrm{ mit } q_i = w : & 0 \\
    \end{array}\right.,\quad
    1 \le i \le ||\vec{M}||
  \end{split}
  \end{equation}
\end{dfn}
\noindent Beispiel: sei $\vec{W} = \{2,\,4,\,5,\,6\}$, dann gilt
$\pi(\vec{W},\,5) = 3$ und $\pi(\vec{W},\,1) = 0$.

Die Funktion $\pi$ kann auch f"ur Werte der unter Definition~\ref{def:lin} auf
Seite~\ref{def:lin} eingef"uhrten Linearisierungsfunktion verwendet werden.


%--------------------------------------
%
% Wichtige Feststellungen
%
%--------------------------------------
\subsection{Wichtige Feststellungen}
Resultierend aus den vorhergehenden Definitionen k"onnen wir eine Reihe von
Feststellungen notieren.

\begin{thm}[Ende einer Terminserie]\label{thm:endOfSeries}
  Das Ende einer Terminserie $t_e$ ergibt sich wie folgt:
  \begin{equation}t_e := s_n\end{equation}
\end{thm}
\begin{proof}[Beweis]
  Nach Definition~\ref{def:numOfElements} ist $n$ die Anzahl der Termine in der
  Serie, d. h. die Serie endet nach $n$ Serienelementen. Da wir nach
  Definition~\ref{def:seriesAsSeries} eine Serie als Folge auf\mbox{}fassen,
  muss das letzte Element $s_n$ sein.
\end{proof}

\begin{thm}[Unechte Terminserie]
  Eine Terminserie mit nur einem Element bezeichnen wir als "`unechte
  Terminserie"'. Es gilt:
  \begin{equation}t'_s = t_e \Leftrightarrow n = 1\end{equation}
\end{thm}
\begin{proof}[Beweis]
  Wir beweisen zuerst $t'_s = t_e \Rightarrow n = 1$. Laut
  Definition~\ref{def:seriesAsSeries} ist $s_1 = t'_s$. Wenn nun gilt
  $t'_s = t_e$, dann muss auch $t_e = s_1$ gelten, d. h. $s_n$ mit $n = 1$.
  \par
  Nun beweisen wir $n = 1 \Rightarrow t'_s = t_e$. Wenn $n = 1$, dann muss
  $s_n = s_1 = t'_s$ gelten nach Definition~\ref{def:seriesAsSeries}. Nun gilt
  aber auch $s_n = s_1 = t_e$ nach Satz~\ref{thm:endOfSeries}. Daraus folgt
  $t'_s = t_e$.
\end{proof}

Die Betrachtung der Serienmuster 20 und 30 ist gleich, wenn man den folgenden
Zusammenhang ber"ucksichtigt:
\begin{equation}
  \label{eqn:20to30}\texttt{* * * 1-5}\ \hat{=}\ \texttt{* * * 1,2,3,4,5/1}
\end{equation}
Wir k"onnen uns damit auf die Betrachtung von Fall 30 beschr"anken.



%================================================
%
%
% Pruefung auf ein Serienelement
%
%
%================================================
\section{Pr"ufung auf ein Serienelement}
\noindent Dieser Abschnitt beschreibt, wie gepr"uft werden kann, ob ein
gegebenens Datum $d$ Element einer Terminserie $\vec{S}$ ist. Dazu wird in
folgender Reihenfolge vorgegangen:
\begin{enumerate}
\item Pr"ufung des Basiswertes jedes Musterfeldes in folgender Reihenfolge:
  \begin{enumerate}
  \item Pr"ufung von einfachen Werten
  \item Pr"ufung von Aufz"ahlungs- und Bereichswerten
  \item Pr"ufung von Relativwerten
  \end{enumerate}
  Anschlie"send sollte jedes Musterfeld durch eine der o. g. Pr"ufungen
  abgedeckt worden sein.
\item Pr"ufung von optionalen Qualifizierern in folgender Reihenfolge:
  \begin{enumerate}
  \item Pr"ufung von Intervallwerten
  \item Pr"ufung von Ordinalwerten
  \end{enumerate}
\end{enumerate}


%--------------------------------------
%
% Pruefung von Basiswerten
%
%--------------------------------------
\subsection{Pr"ufung von Basiswerten}
\subsubsection{Pr"ufung einfacher Werte}
Einfache Werte werden nach folgenden Regeln gepr"uft:
\begin{enumerate}
  \item Ist der Basiswert \texttt{*} passt er auf jeden beliebigen Wert.
  \item Ist der Basiswert der exakte Wert \texttt{\textit{x}}, passt das
    jeweilige Feld von $d \in \Psi$ dann und nur dann, wenn gilt:
    \begin{itemize}
      \item Im Tagesmuster (Muster 40 und 60): $\dayf(d) = x$
      \item Im Monatsmuster (Muster 60 und 70): $\monf(d) = x$
      \item Im Wochentagsmuster (Muster 30, 50 und 70): $\wdf(d) = x$
    \end{itemize}
\end{enumerate}

\subsubsection{Pr"ufung von Aufz"ahlungs- und Bereichswerten}
Nach Satz~\ref{thm:range} wandeln wir zun"achst Bereichswerte in
Aufz"ahlungswerte um und arbeiten hinfort mit einer Menge $\vec{M}$ von Werten.
Das jeweilige Feld von $d \in \Psi$ passt dann und nur dann, wenn gilt:
\begin{itemize}
  \item Im Wochentagsmuster (Muster 30, 50 und 70): $\wdf(d) \in \vec{M}$
\end{itemize}

\subsubsection{Pr"ufung von Relativwerten}
Zun"achst muss das Vorkommen des Ursprungs $d_U \in \Psi$ im Jahr $\yearf(d)$
von $d \in \Psi$ ermittelt werden. Das jeweilige Feld von $d$ passt dann und nur
dann, wenn gilt:
\begin{itemize}
  \item Im Tagesmuster: $d = \calf(\dayf(d_U) + x,\,\monf(d_U),\,\yearf(d_U),\,,\,,)$
  \item Im Monatsmuster: $d = \calf(\dayf(d_U),\,\monf(d_U) + x,\,\yearf(d_U),\,,\,,)$
\end{itemize}


%--------------------------------------
%
% Pruefung von Qualifizierern
%
%--------------------------------------
\subsection{Pr"ufung von Qualifizierern}
Sind Qualifizierer angegeben, werden die gepr"uften Basiswerte damit verkn"upft
und gegen $d \in \Psi$ gepr"uft.

\subsubsection{Pr"ufung von Intervallwerten}
In Abh"angigkeit vom Musterfeld werden die Intervallwerte wie folgt gepr"uft:
\begin{itemize}
\item Tagesmuster. Zun"achst wird die Differenz $\Delta$ zwischen $d$ und $t'_s$
  in Tagen ermittelt:
  \begin{equation}
    \Delta = \left\lfloor\frac{d - t'_s}{1 \textrm{ Tag}}\right\rfloor
  \end{equation}
  Das Datum $d$ passt dann und nur dann wenn gilt:
  \begin{equation}(\Delta \ge 0) \wedge (\Delta \bmod n = 0)\end{equation}
\item Monatsmuster. Nach~\eqref{eqn:monthDiff} wird zun"achst die Differenz
  $\Delta_m$ in Monaten zwischen $d$ und $t'_s$ ermittelt. Das Datum $d$ passt
  dann und nur dann wenn gilt:
  \begin{equation}(\Delta_m \ge 0) \wedge (\Delta_m \bmod n = 0)\end{equation}
\item Jahresmuster. Die Differenz in Jahren $\Delta$ wird zun"achst wie folgt
  errechnet:
  \begin{equation}\Delta = \yearf(d) - \yearf(t'_s)\end{equation}
  Das Datum $d$ passt dann und nur dann wenn gilt:
  \begin{equation}(\Delta \ge 0) \wedge (\Delta \bmod n = 0)\end{equation}
\item Wochentagsmuster. Zun"achst wird die Differenz $\Delta$ zwischen $d$ und
  $t'_s$ in Wochen ermittelt:
  \begin{eqnarray}
    \Delta & = & \frac{d - t'_s}{1 \textrm{ Woche}} \\
    \Delta' & = & \lfloor\Delta\rfloor
  \end{eqnarray}
  Das Datum $d$ passt wenn
  \begin{equation}\label{eqn:weekStepMatch}
    (\Delta' \ge 0) \wedge (\Delta' \bmod n = 0)
  \end{equation}
  \eqref{eqn:weekStepMatch} gilt allerdings nicht unbedingt, wenn $\Delta' = 0$.
  Dieser Fall kann n"amlich sowohl auftreten, wenn $d = t'_s$ gilt (dann passt
  $d$) als auch wenn $0 < \Delta < 1$ gilt (dann liegt $d$ in der ersten Woche
  nach $t'_s$, also zwischen $t'_s$ und $t'_s + 7$ Tage (und $d$ passt dann
  nicht).
\end{itemize}

\subsubsection{Pr"ufung von Ordinalwerten}
Ordinalwerte k"onnen laut Definition nur in den Wochentagsmustern vorkommen.
Hier ist eine Unterscheidung zwischen $w < 0$ und $w \ge 0$ notwendig:
\begin{itemize}
\item $w \ge 0$. $d$ passt dann und nur dann wenn gilt:
  \begin{equation}\womf(d) = w\end{equation}
\item $w < 0$. Hier erstellen wir uns ein verl"aufiges Datum $d'$,
  da $\forall\ d: \womf(d) \ge 0$ gilt.
  \begin{equation}d' := \calf(,\,m',\,\yearf(d),\,ws,\,,\,w)\end{equation}
  wobei wir den Monat $m'$ in Abh"angigkeit vom verwendeten Basiswert des
  Monatsmusters $patt$ abh"angig machen:
  \begin{equation}
  m' := \left\{\begin{array}{ll}
  patt = \texttt{*} : & \monf(d) \\
  patt = \texttt{\textit{x}} : & x \\
  \end{array}\right.
  \end{equation}
  Das Datum $d$ passt dann und nur dann wenn gilt:
  \begin{equation}
    (\dayf(d) = \dayf(d')) \wedge (\monf(d) = \monf(d'))
  \end{equation}
\end{itemize}


%--------------------------------------
%
% Zusammenfassende Formeln
%
%--------------------------------------
\subsection{Zusammenfassende Formeln}\label{ssec:dateInSeries}
Aus den oben aufgef"uhrten Regeln kann f"ur die einzelnen Serientypen $y$ ein
Pr"adikat $P_y$ nach folgenden Formeln berechnet werden, das angibt ob ein Datum
$d \in \Psi$ Element einer Terminserie $\vec{S}$ ist.

\subsubsection{Serie 10}
Gegeben sei die Serie \texttt{*/\textit{n} * * *}.
\begin{equation}
\begin{split}
  \Delta & = \left\lfloor\frac{d - t'_s}{1 \textrm{ Tag}}\right\rfloor \\
  P_{10} & \equiv (\Delta \ge 0) \wedge (\Delta \bmod n = 0)
\end{split}
\end{equation}

\subsubsection{Serie 30}
Gegeben sei die Serie \texttt{* * * \textit{ws}/\textit{n}}.
\begin{equation}
\begin{split}
  \Delta & = \left\lfloor\frac{d - t'_s}{1 \textrm{ Monat}}\right\rfloor \\
  P_{30} & \equiv (\wdf(d) \in ws) \wedge (\Delta \ge 0) \wedge (\Delta \bmod n = 0)
\end{split}
\end{equation}

\subsubsection{Serie 40}
Gegeben sei die Serie \texttt{\textit{dy} */\textit{n} * *}.
\begin{equation}
\begin{split}
  \Delta & = \left\lfloor\frac{d - t'_s}{1 \textrm{ Monat}}\right\rfloor \\
  P_{40} & \equiv (\dayf(d) = dy) \wedge (\Delta \ge 0) \wedge (\Delta \bmod n = 0)
\end{split}
\end{equation}

\subsubsection{Serie 50}
Gegeben sei die Serie \texttt{* */\textit{n} * \textit{ws}:\textit{w}}.
\begin{equation}
\begin{split}
  \Delta & = \left\lfloor\frac{d - t'_s}{1 \textrm{ Monat}}\right\rfloor \\
  d' & = \calf(,\,\monf(d),\,\yearf(d),\,ws,\,,\,w) \\
  P' & \equiv \left\{\begin{array}{ll}
      w \ge 0 & : \womf(d) = w \\
      w < 0 & : \dayf(d) = \dayf(d')
    \end{array}\right. \\
  P_{50} & \equiv (\wdf(d) = ws) \wedge P' \wedge
    (\Delta \ge 0) \wedge (\Delta \bmod n = 0)
\end{split}
\end{equation}

\subsubsection{Serie 60}
Gegeben sei die Serie \texttt{\textit{dy} \textit{m} */1 *}.
\begin{equation}
  P_{60} \equiv (\dayf(d) = dy) \wedge (\monf(d) = m)
\end{equation}

\subsubsection{Serie 70}
Gegeben sei die Serie \texttt{* \textit{m} */1 \textit{ws}:\textit{w}}.
\begin{equation}
\begin{split}
  d' & = \calf(,\,\monf(d),\,\yearf(d),\,ws,\,,\,w) \\
  P' & \equiv \left\{\begin{array}{ll}
      w \ge 0 & : \womf(d) = w \\
      w < 0 & : \dayf(d) = \dayf(d')
    \end{array}\right. \\
  P_{70} & \equiv (\wdf(d) = ws) \wedge (\monf(d) = m) \wedge P'
\end{split}
\end{equation}



%================================================
%
%
% Kalibrierung des Startzeitpunkts
%
%
%================================================
\section{Kalibrierung des Startzeitpunkts}


%--------------------------------------
%
% Notwendigkeit der Kalibrierung
%
%--------------------------------------
\subsection{Notwendigkeit der Kalibrierung}
\noindent Eine Kalibrierung des Startzeitpunkts einer Terminserie muss immer
dann stattfinden, wenn der vom Benutzer gew"ahlte Serienbeginn nicht selbst Teil
der Serie ist. In anderen Worten: es ist eine Kalibrierung notwendig, wenn gilt:
\begin{equation}
  t_s \notin \vec{S}
\end{equation}

Wir definieren ein Pr"adikat $P_y$ f"ur die einzelnen Serientypen $y$, das nach
folgenden Formeln berechnet wird. Es gibt an, ob ein Datum $d \in \Psi$ bereits
kalibriert ist, also das Startdatum der Serie fungieren kann.

Die Pr"adikate ergeben sich aus den Formeln im Abschnitt~\ref{ssec:dateInSeries}
auf Seite~\pageref{ssec:dateInSeries}, wobei die Betrachtung der Intervallwerte
weggelassen wird.

\subsubsection{Serie 10}
Gegeben sei die Serie \texttt{*/\textit{n} * * *}. Offensichtlich ist im Fall 10
keine explizite Kalibrierung notwendig, da gilt:
\begin{equation}\forall\ t_s : t_s \in \vec{S}\end{equation}
Daraus folgt:
\begin{equation}P_{10} \equiv \top\end{equation}

\subsubsection{Serie 30}
Gegeben sei die Serie \texttt{* * * \textit{ws}/\textit{n}}. Eine Kalibrierung
ist notwendig, wenn der Wochentag des gegebenen Datums $t_s$ nicht in der Menge
$ws$ enthalten ist, also:
\begin{equation}P_{30} \equiv \wdf(t_s) \in ws\end{equation}

\subsubsection{Serie 40}
Gegeben sei die Serie \texttt{\textit{dy} */\textit{n} * *}. Eine Kalibrierung
ist notwendig, wenn der Tag des gegebenen Datums $t_s$ ungleich $dy$ ist, also:
\begin{equation}P_{40} \equiv \dayf(t_s) = dy\end{equation}

\subsubsection{Serie 50}
Gegeben sei die Serie \texttt{* */\textit{n} * \textit{ws}:\textit{w}}. Eine
Kalibrierung ist notwendig, wenn der Wochentag des gegebenen Datums $t_s$
ungleich $ws$ ist oder der Ordinalwert der Wochennummer nicht passt, also:
\begin{equation}
\begin{split}
  d' & = \calf(,\,\monf(t_s),\,\yearf(t_s),\,ws,\,,\,w) \\
  P' & \equiv \left\{\begin{array}{ll}
      w \ge 0 & : \womf(t_s) = w \\
      w < 0 & : \dayf(t_s) = \dayf(d')
    \end{array}\right. \\
  P_{50} & \equiv (\wdf(t_s) = ws) \wedge P'
\end{split}
\end{equation}

\subsubsection{Serie 60}
Gegeben sei die Serie \texttt{\textit{dy} \textit{m} */1 *}. Eine Kalibrierung
ist notwendig, wenn der Tag des gegebenen Datums $t_s$ ungleich $dy$ und der
Monat ungleich $m$ ist, also:
\begin{equation}
  P_{60} \equiv (\dayf(t_s) = dy) \wedge (\monf(t_s) = m)
\end{equation}

\subsubsection{Serie 70}
Gegeben sei die Serie \texttt{* \textit{m} */1 \textit{ws}:\textit{w}}. Eine
Kalibrierung ist notwendig, wenn der Wochentag des gegebenen Datums $t_s$
ungleich $ws$ ist, der Ordinalwert der Wochennummer nicht passt oder der Monat
ungleich $m$ ist, also:
\begin{equation}
\begin{split}
  d' & = \calf(,\,\monf(t_s),\,\yearf(t_s),\,ws,\,,\,w) \\
  P' & \equiv \left\{\begin{array}{ll}
      w \ge 0 & : \womf(t_s) = w \\
      w < 0 & : \dayf(t_s) = \dayf(d')
    \end{array}\right. \\
  P_{70} & \equiv (\wdf(t_s) = ws) \wedge (\monf(t_s) = m) \wedge P'
\end{split}
\end{equation}


%--------------------------------------
%
% Formeln fuer die Kalibrierung
%
%--------------------------------------
\subsection{Formeln f"ur die Kalibrierung}
Nachfolgend sind f"ur die Serientypen, die "uberhaupt eine Kalibrierung
erfordern die Formeln zur Berechnung des kalibrierten Startdatums $t'_s$
angegeben.

\subsubsection{Serie 30}
Gegeben sei die Serie \texttt{* * * \textit{ws}/\textit{n}}. Die Kalibrierung im
Fall 30 ist nicht trivial. Zun"achst ist zu "uberpr"ufen, ob sich einer der
Wochentag $ws$ noch in der aktuellen Woche befindet, also die folgende Aussage
wahr ist:
\begin{equation}
  \exists\ \xi \in ws :\ \linf(\wdf(t_s)) < \linf(\xi) \le \eta + T
\end{equation}
Ist sie wahr, ist das zugeh"orige Datum das kalibrierte Startdatum:
\begin{equation}
  t'_s = \calf(,\,,\,\yearf(t_s),\,\xi,\,\weekf(t_s) +
    \left\lfloor\frac{\xi}{T}\right\rfloor,\,)
\end{equation}
Andernfalls muss die Suche nach dem passenden Wochentag nach $n - 1$ Wochen
stattfinden:
\begin{equation}\exists\ \xi \in ws :\ \eta \le \xi < \wdf(t_s)\end{equation}
Dann ergibt sich als kalibriertes Startdatum:
\begin{equation}
  t'_s = \calf(,\,,\,\yearf(t_s),\,\xi,\,\weekf(t_s) + n,\,)
\end{equation}

\subsubsection{Serie 40}
Gegeben sei die Serie \texttt{\textit{dy} */\textit{n} * *}. Im Fall 40 findet
sich das kalibrierte Startdatum wie folgt. Dabei sei
$t_x = \calf(dy,\,\monf(t_s),\,\yearf(t_s),\,,\,,\,)$.
\begin{equation}
  t'_s = \left\{\begin{array}{ll}
      t_s < t_x: & t_x \\
      t_s > t_x: & \addff(\monf,\,t_x,\,n) \\
    \end{array}\right.
\end{equation}

\subsubsection{Serie 50}
"Ahnlich auch im Fall 50. Gegeben sei die Serie
\texttt{* */\textit{n} * \textit{ws}:\textit{w}}. Sei zun"achst der n"achste
infragekommende Termin $t_x$ mit
$t_x = \calf(,\,\monf(t_s),\,\yearf(t_s),\,,\,ws,\,w)$ bezeichnet. Dann gilt
f"ur das kalibrierte Startdatum:
\begin{equation}
  t'_s = \left\{\begin{array}{ll}
      t_s < t_x : & t_x \\
      t_s > t_x : & \addff(\monf,\,t_x,\,n) \\
    \end{array}\right.
\end{equation}

\subsubsection{Serie 60}
Gegeben sei die Serie \texttt{\textit{dy} \textit{m} */1 *}. F"ur Fall 60 gilt
unter Ber"ucksichtigung des n"achsten infragekommenden Termins $t_x$ mit
$t_x = \calf(dy,\,m,\,\yearf(t_s),\,,\,,\,)$:
\begin{equation}
  t'_s = \left\{\begin{array}{ll}
    t_s < t_x : & t_x \\
    t_s > t_x : & \addff(\yearf,\,t_x,\,1) \\
  \end{array}\right.
\end{equation}

\subsubsection{Serie 70}
Gegeben sei die Serie \texttt{* \textit{m} */1 \textit{ws}:\textit{w}}. Hier
berechnet sich das kalibrierte Startdatum f"ur Fall 70 mit dem n"achsten
infragekommenden Termin $t_x$ mit $t_x = \calf(,\,m,\,\yearf(t_s),\,,\,ws,\,w)$
wie folgt:
\begin{equation}
  t'_s = \left\{\begin{array}{ll}
    t_s < t_x : & t_x \\
    t_s > t_x : & \addff(\yearf,\,t_x,\,1) \\
  \end{array}\right.
\end{equation}



%================================================
%
%
% Intervall- und Endzeitpunktberechnung
%
%
%================================================
\section{Intervall- und Endzeitpunktberechnung}
\noindent Wie in Tabelle~\ref{tab:pattern} auf Seite~\pageref{tab:pattern}
dargestellt, ergibt sich f"ur die F"alle 10 und 40 bis 70 ein regelm"a"siges
Intervall $\varepsilon$, und es gilt f"ur die Berechnung des Zeitpunkts $t_e$
nach $c$ Terminen:
\begin{equation}\label{eqn:endEasy}
  t_e = t'_s + \varepsilon(c - 1)
\end{equation}

\noindent Nicht trivial ist dagegen die Berechnung der Intervalle des Zeitpunkts
$t_e$ f"ur den Fall 30. Gegeben sei die Serie
\texttt{* * * \textit{ws}/\textit{n}}. Wir schaffen uns zun"achst eine geordnete
Liste $\vec{\Theta}$ f"ur $ws = \langle w_1,\,w_2,\,\dots,\,w_m\rangle$ wie
folgt:
\begin{equation}\label{eqn:theta}
  \vec{\Theta} = \langle \linf(w_1),\,\linf(w_2),\,\dots,\,\linf(w_m)\rangle
\end{equation}

\noindent Auf unsere transformierte Menge $\vec{\Theta}$ wenden wir eine
Funktion $\Delta$ an, die den Abstand in Tagen zwischen dem Wochentag an der
Stelle $i$ und dem Wochentag an der Stelle $i+1$ ermittelt:
\begin{equation}
\begin{split}
  & \Delta :\, \vec{\mathbb{N}},\,\mathbb{N},\,\mathbb{N} \mapsto \mathbb{N} \\
  & \Delta(\vec{\Theta},\,i,\,n) = \left\{\begin{array}{ll}
  1 \le i < \|\vec{\Theta}\| : & \vec{\Theta}_{i+1} - \vec{\Theta}_i \\
  i = \|\vec{\Theta}\| : & nT + \vec{\Theta}_{1} - \vec{\Theta}_{i} \\
  \end{array}\right.,\,1 \le i \le \|\vec{\Theta}\|
\end{split}
\end{equation}
Die Funktion $\Delta$ berechnet au"serdem den Abstand zwischen dem letzten
Wochentag $\vec{\Theta}_m$ und dem ersten Wochentag $\vec{\Theta}_1$ nach $n$
Wochen.

\noindent Dadurch entsteht eine Reihe $s_n$ von Terminen dieser Serie:
\begin{equation}\label{eqn:computeSeries30}
\begin{split}  
  s_1 & = t'_s \\
  s_{j} & = s_{j-1} + \Delta(\vec{\Theta},\,\pi(\vec{\Theta},\,\linf(\wdf(s_{j-1}))),\,n)
\end{split}
\end{equation}

\noindent F"ur eine Anzahl von $c$ Terminen ergibt sich somit folgender
Endtermin $t_e$ im Falle 30:
\begin{equation}\label{eqn:sn300}t_e = s_c\end{equation}

\noindent Diese Berechnung ist allerdings bei der Implementation teuer, da sie
auf einem induktiven Prinzip basiert und die Laufzeit linear mit $c$ w"achst,
also $O(c)$ gilt. Der folgende Ansatz berechnet $t_e$ in konstanter Laufzeit
$O(1)$.

Sei $k = \|\vec{\Theta}\|$. Wir beobachten, dass jeweils bis zu $k$ Termine in
einer Woche liegen k"onnen. Daraus errechnen wir einen Wert $\varphi$ wie folgt:
\begin{equation}
  \varphi = k \left\lfloor\frac{c - 1}{k}\right\rfloor + 1 \\
\end{equation}
Wir verwenden $c - 1$, da der Startzeitpunkt $t'_s$ zu den $c$ Terminen der
Serie geh"ort, aber nicht in die Berechnung einflie"sen soll. Dann gilt
\begin{equation}
  s_\varphi = t'_s + nT \left\lfloor\frac{c - 1}{k}\right\rfloor\textrm{ Tage}
\end{equation}

\noindent Die Werte $s_{\varphi + 1}$ bis $s_c$ errechnen wir
nach~\eqref{eqn:computeSeries30}. Dadurch ergibt sich eine Laufzeit von
$O(1) + O(c) = O(1)$.



%================================================
%
%
% Approximierung
%
%
%================================================
\section{Approximierung}
Eine wichtige Problemstellung ist die Findung eines Termins innerhalb einer
Serie, der einem gegebenen Datum $d_x$ chronologisch als n"achstes folgt. Die
folgende Definition beschreibt das:
\begin{dfn}[approximierter Zeitpunkt]\label{def:approx}
  Gegeben sei ein Datum $d_x \in \Psi$ und eine Terminserie $s$. Unter der
  Voraussetzung, dass $d_x \le t_e = s_c$, gilt
  \begin{equation}\exists\ \vartheta : \vartheta \ge d_x\end{equation}
  Dabei ist $\vartheta \in \vec{S}$ ein Serienelement $s_i \in \vec{S}$, das
  $d_x$ chronologisch als n"achstes folgt:
  \begin{equation}\label{eqn:approxGeneral}
  \begin{split}
    & \vartheta := s_i \textnormal{ mit } d_x = s_i - \delta,\ 1 \le i \le c
      \textnormal{ und } \delta \ge 0 \\
    & \forall j \textnormal{ mit } i < j \le c :\ s_j - d_x > \delta
  \end{split}
  \end{equation}
  Weiterhin gilt:
  \begin{eqnarray}\label{eqn:approxEasy}
    d_x < t'_s \Longrightarrow \vartheta = t'_s
  \end{eqnarray}
\end{dfn}

\paragraph{Anmerkungen}
\begin{itemize}
  \item Die Voraussetzung $d_x \le t_e = s_c$ ergibt sich daraus, dass
    abgelaufene Termine nie wieder stattfinden werden.
  \item Der Wert $\delta$ bestimmt kleinsten m"oglichen Wert, der, zu $d_x$
    addiert, ein Element $s_i$ der Terminserie ergibt.
\end{itemize}

\noindent Die Ermittlung von $\vartheta$ ist vom Fall der Terminserie abh"angig.
Nachfolgend wird die Berechnung f"ur jeden Fall untersucht. Im
Fall~\eqref{eqn:approxEasy} ist $\vartheta$ bereits bestimmt worden. In den
folgenden Abschnitten wollen wir deshalb von $d_x \ge t'_s$ ausgehen. Dazu
halten wir zun"achst folgendes fest:
\begin{thm}\label{thm:dBetweenElems}
  Unter der Voraussetzung $t'_s \le d_x \le t_e$ gilt
  \begin{equation}
    \exists\ i \in \mathbb{N},\ 1 \le i < c :\ s_i \le d_x \le s_{i+1}
  \end{equation}
  $d_x$ muss sich also zwischen zwei benachbarten Serienelementen befinden.
\end{thm}
\begin{proof}[Beweis]
  Vom Bereich her ist Satz~\ref{thm:dBetweenElems} korrekt, denn es gilt
  \begin{equation}
  \begin{split}
    \forall\ i \in \mathbb{N},\,1 \le i < c : & \ s_i \le d_x \le s_{i+1} \\
    \Longleftrightarrow & \ s_1 \le d_x \le s_c \\
    \Longleftrightarrow & \ t'_s \le d_x \le t_e
  \end{split}
  \end{equation}
  Das wurde durch die Voraussetzung bereits sichergestellt. Nach
  Definition~\ref{def:seriesAsSeries} gilt
  \begin{equation}
  \begin{split}
    s_{i+1} & = s_i + \varepsilon \\
    s_{i+1} - s_i & = \varepsilon
  \end{split}
  \end{equation}
  Nach den Werten f"ur $\varepsilon$ aus Tabelle~\ref{tab:pattern} auf
  Seite~\pageref{tab:pattern} erkennen wir, dass $\varepsilon \ge 1$ Tag gilt,
  d. h. $s_{i+1} - s_i \ge 1$ Tag. Die L"ange eines Datums $d_x$ ist
  offensichtlich 1 Tag. Es gilt also
  \begin{equation}s_i + 1 \textrm{ Tag} \le s_i + \varepsilon\end{equation}
\end{proof}


%--------------------------------------
%
% Serie 10
%
%--------------------------------------
\subsection{Serie 10}
Gegeben sei die Serie \texttt{*/\textit{n} * * *}. Dem Fall 10 liegt als
einzigem Fall ein fester, deterministischer Intervall zugrunde. Wir wollen nun
den Index $q$ desjenigen Serienelements $s_q$ ermitteln, f"ur das gilt
$d_x = s_q - \delta$ mit $\delta \ge 0$ und
$\forall j,\ i < j \le c :\ s_j - d_x > \delta$. Das bedeutet, wir ermitteln den
Index des Serienelements, das $d_x$ chronologisch als n"achstes folgt. Dabei
verwenden wir Gleichung~\eqref{eqn:endEasy} und machen uns zunutze, dass der
dort betrachtete Endzeitpunkt ein Element der Terminserie ist. Wir ersetzen
deshalb in~\eqref{eqn:endEasy} $c$ durch $q$ und $t_e$ durch $d_x$ und stellen
nach $q$ um. Dabei gilt $\varepsilon = n$ Tage.
\begin{equation}\label{eqn:qFromEpsilon}
\begin{split}
  d_x & = t'_s + \varepsilon(q - 1) \\
  d_x - t'_s & = \varepsilon(q - 1) \\
  \frac{d_x - t'_s}{\varepsilon} & = q - 1 \\
  \frac{d_x - t'_s}{\varepsilon} + 1 & = q
\end{split}
\end{equation}
Damit gilt f"ur $\vartheta$:
\begin{equation}\label{eqn:theta10}
\begin{split}
  \vartheta & = t'_s + \varepsilon\lfloor q\rfloor \\
  & = t'_s + \varepsilon\left\lfloor \frac{d_x - t'_s}{\varepsilon} +
    1\right\rfloor
\end{split}
\end{equation}
Diese Formel gilt jedoch nicht, wenn $d_x - t'_s = 0$. In diesem Fall ist $d_x$
bereits ein Serienelement und es darf nicht das n"achste ausgew"ahlt werden.
Mit einem Trick l"osen wir dieses Problem: wenn wir statt $d_x$ den Vortag
$d_x - 1$ verwenden, geh"ort auch ein Serienelement mit zum Intervall. Daher
wandeln wir~\eqref{eqn:theta10} wie folgt ab:
\begin{equation}
  \vartheta = t'_s + \varepsilon\left\lfloor
    \frac{d_x - 1 - t'_s}{\varepsilon} + 1\right\rfloor
\end{equation}


%--------------------------------------
%
% Serie 30
%
%--------------------------------------
\subsection{Serie 30}
Gegeben sei die Serie \texttt{* * * \textit{ws}/\textit{n}}. Wir verwenden
wieder die geordnete Liste $\vec{\Theta}$ wie in~\eqref{eqn:theta} auf
Seite~\pageref{eqn:theta} definiert.

Um $\vartheta$ f"ur den Fall 30 zu berechnen, ist eine umfangreichere
Betrachtung vonn"oten. Da kein regelm"a"siges Intervall zugrundeliegt, m"ussen
wir uns einzig und allein mit der Folge $s_n$ behelfen. Um das Durchlaufen der
Folge zu vermeiden, verwenden wir einen Approximationsalgorithmus, mit dem wir
uns an $\vartheta$ ann"ahern.

Wir erkennen zun"achst, dass sich die Serienelemente aller $n$ Wochen
wiederholen. Wir errechnen erst einmal die Anzahl der Wochen $q$ vom Startdatum.
\begin{equation}q = \left\lfloor\frac{d_x - t'_s}{T}\right\rfloor\end{equation}
Das hei"st, $d_x$ liegt in der $q$. Woche seit $t'_s$. Nun beobachten wir
folgendes:
\begin{itemize}
  \item Wenn $q \bmod n = 0$, dann liegt $d_x$ in einer Serienwoche, d. h. einer
    Woche, die an den Wochentagen in $\vec{\Theta}$ Termine der Serie enth"alt.
  \item Wenn $q \bmod n \ne 0$, dann liegt $d_x$ in eine Zwischenwoche, d. h.
    einer Woche zwischen zwei Serienwochen.
\end{itemize}

\noindent Betrachten wir den ersten Fall ($q \bmod w = 0$). Der Wochentag $w$
des gegebenen Datums $d_x$ ermitteln wir mit
\begin{equation}w = \linf(\wdf(d_x))\end{equation}
Weiterhin sei $k$ die Anzahl der Elemente in $\vec{\Theta}$, also
\begin{equation}k = ||\vec{\Theta}||\end{equation}
Um die Differenz $\delta$ zum n"achsten Wochentag der Serie in Woche $q$ bzw.
der n"achstm"oglichen Woche zu ermitteln, definieren wir eine Hilfsfunktion
$\gamma$ wie folgt:
\begin{equation}
\begin{split}
  & \gamma : \vec{\mathbb{N}},\,\wkdays \mapsto \mathbb{N} \\
  & \gamma(\vec{W},\,w) := i \textrm{ mit } 1 < i \le \|\vec{W}\| :
    \ \vec{W}_{i-1} < w \le \vec{W}_i
\end{split}
\end{equation}
Die Funktion $\gamma$ ermittelt aus einer geordneten Liste von Wochentagen den
Index desjenigen Wochentages der gleich $w$ ist oder $w$ als n"achstes folgt.
Daraus bilden wir nun folgenden Algorithmus:
\begin{equation}
  \delta = \left\{\begin{array}{ll}
      k = 1 : & 0 \\
      k > 1,\,w < \vec{\Theta}_1: & nT \\
      k > 1,\,\vec{\Theta}_1 \le w \le \vec{\Theta}_k: &
        \vec{\Theta}_i - \vec{\Theta}_{1} \textrm{ mit }
        i = \gamma(\vec{\Theta},\,w) \\
      k > 1,\,w > \vec{\Theta}_k: & nT \\
    \end{array}\right.
\end{equation}

\noindent Wenn die Liste der Wochentage $\vec{\Theta}$ nur einen einzigen
Eintrag hat, wird keine Differenz zum n"achsten Wochentag in der Liste
ermittelt. Statt dessen gehen wir $n$ Wochen weiter.

Andernfalls unterscheiden wir drei F"alle. Liegt $w$ vor dem ersten Wochentag in
der Liste, dann gehen wir $n$ Wochen weiter. Dies gilt auch, wenn $w$ hinter dem
letzten Wochentag in der Liste liegt. In allen anderen F"allen ermitteln wir mit
der Funktion $\gamma$ den Index des Wochentages der gleich oder gr"o"ser $w$ ist
und berechnen die Differenz in Tagen zum ersten Wochentag aus der Liste.

Nun ist der Errechnung von $\vartheta$ kein Problem mehr:
\begin{equation}\label{eqn:theta30}\vartheta = t'_s + qT + \delta\end{equation}

\noindent Liegt $d_x$ nicht in einer Serienwoche (Fall $q \bmod w \ne 0$), dann
m"ussen wir lediglich die fehlenden Wochen bis zur n"achsten Serienwoche
addieren:
\begin{equation}
  \delta = nT(\left\lfloor\frac{q}{n}\right\rfloor + 1) - qT
\end{equation}
und errechnen schlie"slich nach~\eqref{eqn:theta30} den Wert von $\vartheta$:
\begin{equation}
  \vartheta = t'_s + nT(\left\lfloor\frac{q}{n}\right\rfloor + 1)
\end{equation}


%--------------------------------------
%
% Faelle 40 und 50
%
%--------------------------------------
\subsection{Serien 40 und 50}\label{ssec:approx40_50}
Gegeben seien die Serien \texttt{\textit{dy} */\textit{n} * *} bzw.
\texttt{* */\textit{n} * \textit{ws}:\textit{w}}. Hier l"asst sich f"ur
$\varepsilon$ kein definitiver Wert festlegen, da die Anzahl der Tage in einem
Monat von 28 bis 31 variieren. Aus diesem Grund kann~\eqref{eqn:qFromEpsilon}
nicht angewendet werden und es muss eine Alternative gesucht werden.

Wir berechnen zun"achst die Differenz $\Delta_m$ zwischen $d_x$ und $t'_s$ in
Monaten:
\begin{equation}\label{eqn:deltaM}
  \Delta_m = 12 (\yearf(d_x)-\yearf(t'_s)) + \monf(d_x) - \monf(t'_s)
\end{equation}
Hier ist jetzt eine Fallunterscheidung notwendig. Wir schaffen uns ein Pr"adikat
$P$, dass wahr ist, wenn $d_x$ in einem Monat liegt, in dem die Serie ein
Vorkommen hat. Es gilt also:
\begin{equation}P \equiv \Delta_m \bmod n = 0\end{equation}
Ist $P$ wahr, berechnen wir das Vorkommen $\vartheta'$ der Serie in diesem
Monat, d. h. im Monat $\monf(t'_s) + \Delta_m$ im Jahr $\yearf(t'_s)$. In Serie
40 ist das:
\begin{equation}\label{eqn:thetaStrich40}
  \vartheta' = \calf(dy,\,\monf(t'_s) + \Delta_m,\,\yearf(t'_s),\,,\,,\,)
\end{equation}
In Serie 50 ist das:
\begin{equation}\label{eqn:thetaStrich50}
  \vartheta' = \calf(,\,\monf(t'_s) + \Delta_m,\,\yearf(t'_s),\,ws,\,,\,w)
\end{equation}
\begin{thm}
  Ist $P$ wahr, d. h. liegt $d_x$ in einem Monat, in der die Serie ein Vorkommen
  hat, muss dieses Vorkommen $\vartheta'$ sein.
\end{thm}
\begin{proof}[Beweis]
  Wir beweisen zun"achst, dass $d_x$ und $\vartheta'$ im selben Monat und Jahr
  liegt. Dazu schaffen wir uns eine Hilfsfunktion $m$, die zu einem gegebenen
  Datum die Anzahl der Monate errechnet:
  \begin{equation}
  \begin{split}
    & m : \Psi \mapsto \mathbb{N} \\
    & m(d) := 12\cdot\yearf(d) + \monf(d)
  \end{split}
  \end{equation}
  Nach~\eqref{eqn:deltaM} gilt
  \begin{equation}
  \begin{split}
    \Delta_m & = 12\,(\yearf(d_x)-\yearf(t'_s)) + \monf(d_x) - \monf(t'_s) \\
      & = 12\cdot\yearf(d_x) - 12\cdot\yearf(t'_s) + \monf(d_x) - \monf(t'_s) \\
      & = 12\cdot\yearf(d_x) + \monf(d_x) - (12\cdot\yearf(t'_s) + \monf(t'_s)) \\
      & = m(d_x) - m(t'_s)
  \end{split}
  \end{equation}
  F"ur Monat und Jahr von $\vartheta'$ gilt
  \begin{equation}
  \begin{split}
    m(\vartheta') & = m(t'_s) + \Delta_m \\
      & = m(t'_s) + m(d_x) - m(t'_s) \\
      & = m(d_x)
  \end{split}
  \end{equation}
  Schlie"slich muss in Serie 40 f"ur den Tag des Vorkommens offensichtlich
  gelten:
  \begin{equation}\dayf(\vartheta') = \dayf(t'_s)\end{equation}
  In Serie 50 gilt dagegen:
  \begin{equation}
    \wdf(\vartheta') = \wdf(t'_s) \textrm{ und } \womf(\vartheta') = \womf(t'_s)
  \end{equation}
  Das ergibt sich nach~\eqref{eqn:thetaStrich40} bzw.~\eqref{eqn:thetaStrich50}.
\end{proof}

Wenn $P$ wahr ist, kommt es darauf an, ob $d_x$ chronologisch vor oder hinter
dem Vorkommen $\vartheta'$ der Serie liegt. Gilt $d_x \le \vartheta'$, ist
offensichtlich $\vartheta'$ selbst das gesuchte Vorkommen. Gilt jedoch
$d_x > \vartheta'$, liegt das gesuchte Vorkommen $n$ Monate sp"ater, also zum
Zeitpunkt
\begin{equation*}\vartheta = \addff(\monf,\,\vartheta',\,n)\end{equation*}

Es bleibt noch der Fall $P$ ist falsch, d. h. $d_x$ liegt nicht in einem Monat,
in dem die Serie ein Vorkommen hat.
Dann ergibt sich $\vartheta$ wie folgt:
\begin{equation}\vartheta = \addff(\monf,\,\vartheta',\,n-\Delta_m)\end{equation}

\noindent Zusammenfassend gilt f"ur die Serien 40 und 50 unter Zuhilfenahme
von $\vartheta'$ aus~\eqref{eqn:thetaStrich40} und~\eqref{eqn:thetaStrich50}:
\begin{equation}
  \vartheta = \left\{\begin{array}{ll}
      P,\,d_x \le \vartheta' : & \vartheta' \\
      P,\,d_x > \vartheta' : & \addff(\monf,\,\vartheta',\,n) \\
      \overline{P} : & \addff(\monf,\,\vartheta',\,n-\Delta_m) \\
    \end{array}\right.
\end{equation}


%--------------------------------------
%
% Serien 60 und 70
%
%--------------------------------------
\subsection{Serien 60 und 70}
Gegeben seien die Serien \texttt{\textit{dy} \textit{m} */1 *} bzw.
\texttt{* \textit{m} */1 \textit{ws}:\textit{w}}. Wie schon in der Betrachtung
der Serien 40 und 50 (siehe Seite~\pageref{ssec:approx40_50}) kann auch in
diesen F"allen kein definitiver Wert f"ur das Interval $\varepsilon$ angenommen
werden, da die L"ange eines Jahres zwischen 365 und 366 Tagen variiert. Auch die
Verwendung der genauen Dauer eines tropischen Erdjahres mit
$\varepsilon \approx 365,242198$ Tage ist f"ur kurze Zeitr"aume, die kein
Schaltjahr enthalten, nicht geeignet.

Wir errechnen zun"achst das Datum $\vartheta'$ des Serienvorkommens im aktuellen
Jahr, d. h. im Jahr $\yearf(d_x)$. F"ur Serie 60 gilt
\begin{equation}\vartheta' = \calf(dy,\,m,\,\yearf(d_x),\,,\,,\,)\end{equation}
In Serie 70 gilt
\begin{equation}\vartheta' = \calf(,\,m,\,\yearf(d_x),\,ws,\,,\,w)\end{equation}
Analog zu den Serien 40 und 50 ergibt sich f"ur $\vartheta$ in den Serien 60
und 70:
\begin{equation}
  \vartheta = \left\{\begin{array}{ll}
    d_x \le \vartheta' : & \vartheta' \\
    \vartheta' < d_x : & \addff(\yearf,\,\vartheta',\,1) \\
    \end{array}\right.
\end{equation}

\end{document}
