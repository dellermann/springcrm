%
% Saldierung.tex
%
% Copyright (c) 2011-2013, Daniel Ellermann
%
% This program is free software: you can redistribute it and/or modify
% it under the terms of the GNU General Public License as published by
% the Free Software Foundation, either version 3 of the License, or
% (at your option) any later version.
%
% This program is distributed in the hope that it will be useful,
% but WITHOUT ANY WARRANTY; without even the implied warranty of
% MERCHANTABILITY or FITNESS FOR A PARTICULAR PURPOSE.  See the
% GNU General Public License for more details.
%
% You should have received a copy of the GNU General Public License
% along with this program.  If not, see <http://www.gnu.org/licenses/>.
%


%================================================
%
%
% Praeambel
%
%
%================================================
\documentclass[a4paper]{article}

%--------------------------------------
%
% Packages
%
%--------------------------------------
\usepackage{amsmath}
\usepackage{amssymb}
\usepackage{amsthm}
\usepackage{ngerman}
\usepackage{tabularx}


%--------------------------------------
%
% Kommandos
%
%--------------------------------------
\renewcommand{\theequation}{\thesection.\arabic{equation}}
\numberwithin{equation}{section}
\newtheorem{dfn}{Definition}
\newtheorem{thm}{Satz}
\newtheorem{lmm}{Lemma}


%--------------------------------------
%
% Laengen
%
%--------------------------------------
\newlength{\widthTab}  % Breite von bestimmten Tabellen


%--------------------------------------
%
% Titel
%
%--------------------------------------
\title{Saldierung von Fakturavorg"angen}
\author{Daniel Ellermann}
\date{17. April 2013}


%--------------------------------------
%
% Dokumentstart
%
%--------------------------------------
\begin{document}
\maketitle
\tableofcontents



%================================================
%
%
% Betrifft
%
%
%================================================
\section{Betrifft}
Dieses Dokument beschreibt die Berechnung von Salden von Rechnungen,
Gutschriften und Mahnungen in SpringCRM. Es liefert wichtige Formeln f"ur die
Berechnung und eine Beschreibung der Herleitung dieser Formeln.



%================================================
%
%
% Definitionen
%
%
%================================================
\section{Definitionen}
Zun"achst definieren wir folgende Werte:
\begin{dfn}[Rechnungsbetr"age]
Die Summe einer Rechnung (einschlie"slich MwSt.) bezeichnen wir mit $\sigma_r$.

Den Zahlungsbetrag einer Rechnung, d. h. den durch den Kunden bereits bezahlten
Betrag, bezeichnen wir mit $b_r$.
\end{dfn}
\begin{dfn}[Gutschriftsbetr"age]
Die Summe einer Gutschrift (einschlie"slich MwSt.) bezeichnen wir mit
$\sigma_g$.

Den Zahlungsbetrag einer Gutschrift, d. h. den durch uns bereits bezahlten
Betrag, bezeichnen wir mit $b_g$.
\end{dfn}

\noindent Im folgenden gelten alle Definitionen und Berechnungen von Mahnungen 
wie f"ur Rechnungen, da eine Mahnung eine Art anschlie"sende Rechnung darstellt.



%================================================
%
%
% Salden und Endsalden
%
%
%================================================
\section{Salden und Endsaldo}
\subsection{Salden}
Zu jeder Rechnung und jeder Gutschrift kann ein Saldo errechnet werden. Wir
berechnen den Saldo so, dass ein positiver Saldo (Werte gr"o"ser als null) ein
Guthaben f"ur uns (bzw. eine Verbindungkeit unserseits) und ein negativer Saldo 
(Werte kleiner als null) eine Forderung an den Kunden darstellt. Ein Saldo von
null stellt eine ausgeglichene Forderung dar.

\noindent F"ur Rechnungen ergibt sich somit der Saldo $s_r$ wie folgt:
\begin{equation}s_r := b_r - \sigma_r\end{equation}

\noindent Damit ergibt sich, dass eine nicht oder teilweise bezahlte Rechnung 
einen negativen Saldo aufweist, also $s_r < 0$. Hat der Kunde die Rechnung 
vollst"andig bezahlt, gilt $s_r = 0$.

Der Saldo einer Gutschrift $s_g$ wird genau umgekehrt berechnet:
\begin{equation}s_g := \sigma_g - b_g\end{equation}

\noindent Hier ergibt sich bei einer nicht oder teilweise bezahlten Gutschrift
ein positiver Saldo, also $s_g > 0$. Wurde die Gutschrift vollst"andig bezahlt,
gilt $s_g = 0$.


\subsection{Endsaldo}
F"ur eine abschlie"sende Betrachtung der Salden m"ussen jedoch die Salden der
Rechnung den Salden von eventuellen Gutschriften gegen"ubergestellt werden. Dies
funktioniert wie die Darstellung in einer Bilanz. Wir berechnen das im Endsaldo
$s$, der f"ur Rechnungen und Gutschriften gleich ist:
\begin{equation}s := s_r + \sum s_g\end{equation}

\noindent Es gelten folgende Zusammenh"ange:
\begin{itemize}
  \item Positiver Endsaldo $s > 0$. Der Kunde hat ein Guthaben bei uns, d. h.
    wir schulden ihm Geld.
  \item Negativer Endsaldo $s < 0$. Der Kunde schuldet uns Geld, d. h. wir haben
    Forderungen an ihn.
  \item Endsaldo $s = 0$. Die Rechnung und event. Gutschriften sind 
    ausgeglichen.
\end{itemize}


\subsection{Beispiele}
Ein Kunde kauft Artikel "uber 1.000 EUR. Dann gibt es folgende Beispielf"alle:
\begin{itemize}
  \item Er hat die Rechnung nicht bezahlt. Dann gilt $s_r = 0 - 1000 = -1000$
    und $s = -1000$.
  \item Er bezahlt 800 EUR. Dann gilt $s_r = 800 - 1000 = -200$ und $s = -200$.
  \item Er bezahlt die komplette Rechnung, d. h. 1.000 EUR. Dann gilt
    $s_r = 1000 - 1000 = 0$ und $s = 0$.
\end{itemize}

\noindent Zu dieser Rechnung gibt es zwei Gutschriften "uber 100 EUR und 200
EUR. Dann gibt es folgende Beispielf"alle:
\begin{itemize}
  \item Er hat die Rechnung nicht bezahlt und die Gutschriften wurden auch nicht
    bezahlt. Dann gilt $s_r = 0 - 1000 = -1000$, $s_{g1} = 100 - 0 = 100$,
    $s_{g2} = 200 - 0 = 200$ sowie $s = -1000 + 100 + 200 = -700$. Der Kunde
    muss also noch 700 EUR bezahlen.
  \item Er bezahlt 400 EUR, die Gutschriften wurden nicht bezahlt. Dann gilt
    $s_r = 400 - 1000 = -600$, $s_{g1} = 100 - 0 = 100$, 
    $s_{g2} = 200 - 0 = 200$ sowie $s = -600 + 100 + 200 = -300$. Der Kunde muss
    also noch 300 EUR bezahlen.
  \item Er bezahlt 700 EUR, die Gutschriften wurden nicht bezahlt. Dann gilt
    $s_r = 700 - 1000 = -300$, $s_{g1} = 100 - 0 = 100$, 
    $s_{g2} = 200 - 0 = 200$ sowie $s = -300 + 100 + 200 = 0$. Die Rechnung
    wurde damit vollst"andig bezahlt.
  \item Er bezahlt 900 EUR, die Gutschriften wurden nicht bezahlt. Dann gilt
    $s_r = 900 - 1000 = -100$, $s_{g1} = 100 - 0 = 100$, 
    $s_{g2} = 200 - 0 = 200$ sowie $s = -100 + 100 + 200 = 200$. Wir m"ussen
    noch 200 EUR an den Kunden auszahlen.
  \item Er bezahlt 900 EUR, die Gutschrift "uber 100 EUR wurde bezahlt, die
    andere Gutschrift nicht. Dann gilt $s_r = 900 - 1000 = -100$, 
    $s_{g1} = 100 - 100 = 0$, $s_{g2} = 200 - 0 = 200$ sowie 
    $s = -100 + 0 + 200 = 100$. Wir m"ussen noch 100 EUR an den Kunden 
    auszahlen.
  \item Er bezahlt 1.000 EUR, die Gutschriften wurden bezahlt. Dann gilt 
    $s_r = 1000 - 1000 = 0$, $s_{g1} = 100 - 100 = 0$, $s_{g2} = 200 - 200 = 0$ 
    sowie $s = 0 + 0 + 0 = 0$. Die Rechnung und die Gutschriften wurde damit 
    vollst"andig bezahlt.
\end{itemize}



%================================================
%
%
% Modifizierter Endsaldo
%
%
%================================================
\section{Modifizierter Endsaldo}
Die Formulare f"ur Rechnungen und Gutschriften sollen den noch zu zahlenden
Betrag dynamisch anzeigen k"onnen, d. h. bei "Anderungen der Rechnungs- bzw.
Gutschriftenposten oder des Zahlungsbetrages soll sich eine Angabe "`noch zu
zahlen"' dynamisch neu berechnen. Dies wird mit JavaScript erledigt.

Um die dynamische Berechnung zu vereinheitlichen muss der JavaScript-Funktion
ein normierter Endsaldo mitgegeben werden, der sog. "`modifizierte Endsaldo"'.
Der modifizierte Endsaldo muss zwei Funktionen erf"ullen:
\begin{itemize}
  \item Der Saldo der Rechnung bzw. Gutschrift muss vom Endsaldo subtrahiert
    werden damit die JavaScript-Funktion den Rechnungs- bzw. Gutschriftensaldo
    dynamisch hinzuaddieren kann.
  \item Das Vorzeichen des modifizierten Endsaldos muss geeignet gesetzt sein.
    Bei Rechnungen ist es positiv, damit vom Kunden zu zahlende Betr"age
    positiv erscheinen. Bei Gutschriften ist es dagegen negativ, damit von uns
    zu zahlende Betr"age positiv erscheinen.
\end{itemize}

\noindent Unter diesen Gesichtspunkten definieren wir den modifizierten Endsaldo 
f"ur Rechnungen $m_r$ wie folgt:
\begin{equation}m_r := s - s_r\end{equation}

\noindent Den modifizierten Endsaldo f"ur Gutschriften $m_g$ definieren wir
dagegen wie folgt:
\begin{equation}m_g := -(s - s_g) = s_g - s\end{equation}
Durch das negative Vorzeichen erf"ullen wir das zweite oben genannte Kriterium.
In beiden F"allen wird vom Endsaldo $s$ der Rechnungs- bzw. Gutschriftssaldo
abgezogen (erstes Kriterium), damit ihn die JavaScript-Funktion f"ur den "`noch
zu zahlenden Betrag"' (s. Abschnitt~\ref{sec:stillUnpaid} auf 
Seite~\pageref{sec:stillUnpaid}) dynamisch aufaddieren kann.



%================================================
%
%
% Noch zu zahlender Betrag
%
%
%================================================
\section{Noch zu zahlender Betrag}\label{sec:stillUnpaid}
Wie oben beschrieben, wurde im modifizierten Endsaldo der Rechnungs- bzw. 
Gutschriftssaldo vom Endsaldo abgezogen. Die JavaScript-Funktion kann nun den 
Wert "`noch zu zahlen"' basierend auf dem modifizierten Endsaldo und aktuellen 
Werten f"ur Rechnungssumme und Zahlungsbetrag berechnen. Die aktuelle 
Rechnungssumme bezeichnen wir mit $\sigma'_r$ und den aktuellen Zahlungsbetrag 
mit $b'_r$. Die Differenz zur Rechnungssumme $\sigma_r$ bezeichnen wir mit 
$\Delta\sigma_r$ mit
\begin{equation}\Delta\sigma_r := \sigma'_r - \sigma_r\end{equation}
Weiterhin bezeichnen wir die Differenz zwischen aktuellem Zahlungsbetrag $b'_r$
und Zahlungsbetrag $b_r$ mit $\Delta b_r$ mit
\begin{equation}\Delta b_r := b'_r - b_r\end{equation}
Beim Betrag "`noch zu zahlen"' muss zwei Kriterien erf"ullen:
\begin{itemize}
  \item Das Vorzeichen muss umgekehrt werden, damit negative Salden, die ja 
    eine Forderung an den Kunden bedeuten, als positive "`Noch zu zahlen"'-Werte
    erscheinen.
  \item Der Wert ergibt sich durch Addition des aktuellen Rechnungssaldos $s'_r$
    mit $s'_r := b'_r - \sigma'_r$ bzw. Subtraktion des aktuellen 
    Gutschriftssaldos $s'_g$ mit $s'_g := \sigma'_g - b'_g$ zum bzw. vom
    modifizierten Endsaldo $m_r$ bzw. $m_g$. Durch Addition bzw. Subtraktion der
    \emph{aktuellen} Werte zeigt der "`noch zu zahlende"' Betrag immer den 
    aktuellen offenen Zahlbetrag an.
\end{itemize}

\noindent F"ur Rechnungen ergibt sich die Angabe "`noch zu zahlen"' $z_r$ unter
Beachtung der genannten Kriterien wie folgt:
\begin{equation}
\begin{split}
  z_r := & \,-(m_r + s'_r) \\
    = & \,-(s - s_r + s'_r) \\
    = & \,-(s - b_r + \sigma_r + b'_r - \sigma'_r) \\
    = & \,-(s + b'_r - b_r - \sigma'_r + \sigma_r) \\
    = & \,-(s + \Delta b_r - \Delta\sigma_r) \\
    = & \,\Delta\sigma_r - \Delta b_r - s
\end{split}
\end{equation}
Im speziellen Fall $\sigma'_r = \sigma_r$ und $b'_r = b_r$, also 
$\Delta\sigma_r = \Delta b_r = 0$ gilt:
\begin{equation}z_r = -s\end{equation}

\noindent F"ur Gutschriften ergibt sich die Berechnung analog. Hier bezeichnen
wir die aktuelle Gutschriftssumme mit $\sigma'_g$ und den aktuellen 
Zahlungsbetrag mit $b'_g$. Die Differenz zwischen aktueller Gutschriftssumme
$\sigma'_g$ und Gutschriftssumme $\sigma_g$ bezeichnen wir mit $\Delta\sigma_g$
mit
\begin{equation}\Delta\sigma_g := \sigma'_g - \sigma_g\end{equation}
Weiterhin bezeichnen wir die Differenz zwischen aktuellem Zahlungsbetrag $b'_g$
und Zahlungsbetrag $b_g$ mit $\Delta b_g$ mit
\begin{equation}\Delta b_g := b'_g - b_g\end{equation}
Dann ergibt sich die Angabe "`noch zu zahlen"' $z_g$ f"ur Gutschriften wie
folgt:
\begin{equation}
\begin{split}
  z_g := & \,-(m_g - s'_g) \\
    = & \,-(-(s - s_g) - s'_g) \\
    = & \,-(s_g - s - s'_g) \\
    = & \,-(\sigma_g - b_g - s - \sigma'_g + b'_g) \\
    = & \,-(b'_g - b_g - \sigma'_g + \sigma_g - s) \\
    = & \,-(\Delta b_g - \Delta\sigma_g - s) \\
    = & \,\Delta\sigma_g - \Delta b_g + s
\end{split}
\end{equation}
Auch hier gilt im speziellen Fall $\sigma'_g = \sigma_g$ und $b'_g = b_g$, also 
$\Delta\sigma_g = \Delta b_g = 0$:
\begin{equation}z_g = s\end{equation}



%================================================
%
%
% Saldenausgleich
%
%
%================================================
\section{Saldenausgleich}
Der Benutzer soll durch Klick auf "`noch zu zahlen"' einen Saldenausgleich
herbeif"uhren k"onnen. Dadurch soll der Zahlungsbetrag einer Rechnung bzw.
Gutschrift berechnet werden, dass sich ein ausgeglichener Endsaldo $s = 0$
ergibt.
Der Zahlungsbetrag $c_r$, der bei Klick auf "`noch zu zahlen"' bei einer 
Rechnung einzutragen ist, ergibt sich wie folgt:
\begin{equation}
\begin{split}
  c_r := & \,\sigma_r - m_r \\
    = & \,\sigma_r - s + s_r \\
    = & \,\sigma_r - s + b_r - \sigma_r \\
    = & \,b_r - s \\
    = & \,b_r - s_r - \sum s_g \\
    = & \,b_r - b_r + \sigma_r - \sum s_g \\
    = & \,\sigma_r - \sum s_g 
\end{split}
\end{equation}
Nach Klick auf "`noch zu zahlen"' ist der Zahlungsbetrag also so hoch wie die
Rechnungssumme abz"uglich eventueller Gutschriftssalden. Dadurch ergibt sich
ein ausgeglichener Endsaldo, wenn wir $c_r$ f"ur $b_r$ einsetzen:
\begin{equation}
\begin{split}
  s = & \,s_r + \sum s_g \\
    = & \,c_r - \sigma_r + \sum s_g \\
    = & \,\sigma_r - \sum s_g - \sigma_r + \sum s_g \\
    = & \,0
\end{split}
\end{equation}
Bei Gutschriften erfolgt die Berechnung analog. Der Zahlungsbetrag $c_{g_i}$, 
der bei Klick auf "`noch zu zahlen"' bei einer Gutschrift Nr. $i$ einzutragen 
ist, ergibt sich wie folgt:
\begin{equation}
\begin{split}
  c_{g_i} := & \,\sigma_{g_i} - m_{g_i} \\
    = & \,\sigma_{g_i} - s_{g_i} + s \\
    = & \,\sigma_{g_i} - \sigma_{g_i} + b_{g_i} + s \\
    = & \,b_{g_i} + s \\
    = & \,b_{g_i} + s_r + \sum s_g \\
    = & \,b_{g_i} + s_r + \sum \sigma_g - \sum b_g \\
    = & \,b_{g_i} + s_r + \sum \sigma_g - b_{g_1} - \ldots - b_{g_n} \\
    = & \,s_r + \sum \sigma_g - b_{g_1} - \ldots - b_{g_{i-1}} - b_{g_{i+1}} - 
      \ldots - b_{g_n} \\
\end{split}
\end{equation}
Setzt man den berechneten Zahlungsbetrag $c_{g_i}$ f"ur den Zahlungsbetrag
$b_{g_i}$ der Gutschrift Nr. $i$ ein, ergibt sich ein ausgeglichener Endsaldo
wie folgt:
\begin{equation}
\begin{split}
  s = & \,s_r + \sum s_g \\
    = & \,s_r + \sum \sigma_g - \sum b_g \\
    = & \,s_r + \sum \sigma_g - b_{g_1} - \ldots - b_{g_{i-1}} - \\
      & \,(s_r + \sum \sigma_g - b_{g_1} - \ldots - b_{g_{i-1}} - b_{g_{i+1}} - 
      \ldots - b_{g_n}) - \\
      & \,b_{g_{i+1}} - \ldots - b_{g_n} \\
    = & \,s_r + \sum \sigma_g - b_{g_1} - \ldots - b_{g_{i-1}} - \\
      & \,s_r - \sum \sigma_g + b_{g_1} + \ldots + b_{g_{i-1}} + b_{g_{i+1}} + 
      \ldots + b_{g_n} - \\
      & \,b_{g_{i+1}} - \ldots - b_{g_n} \\
    = & \,0
\end{split}
\end{equation}

\end{document}
